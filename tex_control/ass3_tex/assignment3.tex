\documentclass{article}
\usepackage{graphicx}
\usepackage{adjustbox}
\usepackage{natbib}
\usepackage{hyperref}
\usepackage{xurl}
\usepackage{geometry}
\usepackage{amsmath}
\usepackage{booktabs}
\usepackage{fancyhdr}
\usepackage{setspace}
\usepackage{listings}
\usepackage{xcolor}
\usepackage{float}
\usepackage{subfig}
\geometry{margin=1in}

% Define MATLAB style for listings
\lstdefinestyle{MatlabStyle}{
  language=Matlab,
  basicstyle=\ttfamily\footnotesize,
  keywordstyle=\color{blue},
  stringstyle=\color{red},
  commentstyle=\color{green!50!black},
  numbers=left,
  numberstyle=\tiny\color{gray},
  stepnumber=1,
  numbersep=5pt,
  backgroundcolor=\color{white},
  showspaces=false,
  showstringspaces=false,
  showtabs=false,
  frame=single,
  tabsize=2,
  captionpos=b,
  breaklines=true,
  breakatwhitespace=false,
  escapeinside={\%*}{*)}
}

\begin{document}

  \begin{titlepage}
    \thispagestyle{empty}
    \pagenumbering{Roman}
    \frenchspacing

    \noindent
    \begin{tabular}{@{} c @{\hspace{0.1cm}} c @{\hspace{0.2cm}} c @{}}
      \adjustbox{valign=c}{%
        \includegraphics[height=2cm, trim=0 0 0 0, clip]{sedes.pdf}%
      } &
      \adjustbox{valign=c}{%
        \rule{0.5pt}{2cm}%
      } &
      \adjustbox{valign=c}{%
        \includegraphics[height=2cm, trim=0 0 0 0, clip]{logoFirW.jpg}%
      }
    \end{tabular}
    \hfill

    \vspace*{2cm}

    \begin{center}
      \begin{minipage}[t]{\textwidth}
        \begin{center}
          \large\textbf{B-KUL-H04X3A: Control Theory}
        \end{center}
        \vspace{1cm}
        \begin{center}
          \textbf{Team members:}\\[2mm]
          Lefebure Tiebert (r0887630)\\
          Campaert Lukas (r0885501)\\
        \end{center}
        \vspace{1cm}
        \begin{center}
          \Huge\textbf{Assignment 3: State Feedback and State Estimation}
        \end{center}
        \vspace{1cm}
        \begin{center}
          \underline{Professor:}\\[2mm]
          Prof. Dr. Ir. Jan Swevers\\
        \end{center}
        \vspace{2cm}
        \makebox[\textwidth]{Academic Year 2025-2026}
      \end{minipage}
    \end{center}
  \end{titlepage}

  \newpage
  \vspace*{3.2cm}\vfill

  \begin{center}
    \Large\textbf{\textit{Declaration of Originality}}
  \end{center}

  \noindent \textit{We hereby declare that this submitted draft is entirely our own, subject to feedback and support given us by the didactic team, and subject to lawful cooperation which was agreed with the same didactic team. Regarding this draft, we also declare that:}

  \begin{enumerate}
    \item \textit{Note has been taken of the text on academic integrity \url{https://eng.kuleuven.be/studeren/masterproef-en-papers/documenten/20161221-academischeintegriteit-okt2016.pdf}.}
    \item \textit{No plagiarism has been committed as described on \url{https://eng.kuleuven.be/studeren/masterproef-en-papers/plagiaat\#Definitie:\%20wat\%20is\%20plagiaat?}.}
    \item \textit{All experiments, tests, measurements, \ldots, have been performed as described in this draft, and no data or measurement results have been manipulated.}
    \item \textit{All sources employed in this draft --- including internet sources --- have been correctly referenced.}
  \end{enumerate}

  \newpage

  \section{State estimator and state feedback controller design using pole placement}

  The goal of this assignment is to control the position of the cart along a straight line. The state is the cart position $x(t)$ [m] along the $x$-axis, with the wall at the origin $x=0$. The cart is positioned in front of the wall, so $x<0$. The infrared sensor measures the positive distance to the wall, i.e.\ it measures $-x$.\newline
  
 \noindent The system parameters are:
  \begin{itemize}
    \item Wheel radius: $r = 0.033$~m
    \item Sampling time: $T_s = 0.01$~s (100~Hz control loop)
  \end{itemize}

  
  \noindent The input $u(t) = \omega(t)$ [rad/s] is the common wheel angular velocity setpoint, applied equally to both motors via the inner velocity controllers designed in Assignment~2. The velocity control loop is assumed ideal, meaning the actual wheel velocity tracks the setpoint perfectly.

  \subsection{Discrete-time state equation} % Section 1(a)
  
  \paragraph{Continuous-time state equation:} 
  The velocity control loop from Assignment~2 is assumed ideal, so the actual cart velocity equals the commanded velocity. The continuous-time kinematic relation is:
  \begin{equation}
    \dot{x}(t) = v(t) = r \cdot \omega(t) = r \cdot u(t),
  \end{equation}
  where $v(t)$ [m/s] is the linear cart velocity and $u(t) = \omega(t)$ [rad/s] is the wheel angular velocity input.\\
  
  \noindent In state-space form with scalar state $x(t)$ [m] and input $u(t)$ [rad/s]:
  \begin{equation}
    \dot{x}(t) = A_c x(t) + B_c u(t), \quad A_c = 0,\quad B_c = r = 0.033\ \text{m}.
  \end{equation}

  \paragraph{Forward Euler discretization:} 
  The forward Euler method approximates the derivative as:
  \begin{equation}
    \dot{x}[k] \approx \frac{x[k+1] - x[k]}{T_s}.
  \end{equation}
  Substituting into the continuous-time equation:
  \begin{equation}
    \frac{x[k+1] - x[k]}{T_s} = A_c x[k] + B_c u[k] = 0 \cdot x[k] + r \cdot u[k].
  \end{equation}
  Solving for $x[k+1]$:
  \begin{equation}
    x[k+1] = x[k] + T_s \cdot r \cdot u[k].
  \end{equation}
  
  \noindent The discrete-time state-space model is:
  \begin{equation}
    x[k+1] = A_d x[k] + B_d u[k],
  \end{equation}
  with the discrete-time system matrices:
  \begin{equation}
    \boxed{A_d = 1, \qquad B_d = T_s \cdot r = 0.01 \times 0.033 = 3.3 \times 10^{-4}\ \text{m/(rad/s)}.}
  \end{equation}
  
  \noindent Note: In the state estimator implementation, the measured average wheel speed $u[k] = (\omega_A[k] + \omega_B[k])/2$ is used for the prediction step.


  \subsection{Measurement equation} % Section 1(b)

  The front infrared (IR) sensor measures the distance from the cart to the wall. Since the cart is at position $x < 0$ (in front of the wall at $x = 0$), the measured distance is a positive quantity equal to $|x| = -x$.
  
  \noindent The measurement equation (ignoring noise) is:
  \begin{equation}
    y[k] = -x[k].
  \end{equation}
  
  \noindent In state-space form:
  \begin{equation}
    y[k] = C x[k] + D u[k],
  \end{equation}
  \noindent with the output matrices:
  \begin{equation}
    \boxed{C = -1, \qquad D = 0.}
  \end{equation}
  
  \noindent Here $C = -1$ because the sensor measures the negation of the state, and $D = 0$ since the input does not directly affect the measurement.


  \subsection{Design of state feedback controller gain $K$ using pole placement} % Section 1(c)
  
  The position controller outputs the desired wheel angular velocity based on the position error. Assuming full state feedback (no estimator), the control law is:
  \begin{equation}
    u[k] = K \cdot (x_\text{ref}[k] - x[k]),
  \end{equation}
  where $x_\text{ref}[k]$ [m] is the reference position, $x[k]$ [m] the measured position, $K$ [rad/(s$\cdot$m)] the state feedback gain, and $u[k]$ [rad/s] the commanded wheel angular velocity.

  \paragraph{Derivation of closed-loop pole as function of $K$:}
  Substituting the control law into the discrete-time state equation:
  \begin{align}
    x[k+1] &= A_d x[k] + B_d u[k] \\
           &= x[k] + T_s r \cdot K (x_\text{ref}[k] - x[k]) \\
           &= (1 - T_s r K) x[k] + T_s r K \cdot x_\text{ref}[k].
  \end{align}
  
  \noindent The closed-loop system has the form:
  \begin{equation}
    x[k+1] = A_\text{cl} x[k] + B_\text{cl} x_\text{ref}[k],
  \end{equation}
  where the closed-loop system matrix is:
  \begin{equation}
    \boxed{A_\text{cl} = 1 - T_s r K.}
  \end{equation}
  
  \noindent For this first-order system, the closed-loop pole equals the system matrix:
  \begin{equation}
    \boxed{z_\text{cl}(K) = 1 - T_s r K = 1 - 3.3 \times 10^{-4} \cdot K.}
  \end{equation}

  \paragraph{Pole behavior as function of $K$:}
  At $K = 0$, $z_\text{cl} = 1$ (marginally stable integrator). As $K$ increases, the pole moves left along the real axis, reaching $z_\text{cl} = 0$ (deadbeat) at $K \approx 3030$ rad/(s$\cdot$m) and the stability boundary $z_\text{cl} = -1$ at $K \approx 6060$ rad/(s$\cdot$m).

  \paragraph{Stability analysis:}
  For discrete-time stability, the pole must lie inside the unit circle:
  \begin{equation}
    |z_\text{cl}(K)| = |1 - T_s r K| < 1.
  \end{equation}
  This yields the stability condition:
  \begin{equation}
    \boxed{0 < K < \frac{2}{T_s r} = \frac{2}{0.01 \times 0.033} \approx 6060\ \text{rad/(s$\cdot$m)}.}
  \end{equation}
  
  \noindent\textbf{Yes, the system can become unstable} if $K > 6060$ rad/(s$\cdot$m), causing the pole to exit the unit circle through $z = -1$.

  \paragraph{Pole-zero map:}
  Figure~\ref{fig:pole_map_K} shows the closed-loop pole location for varying $K$. The pole moves along the real axis from $z = 1$ (at $K = 0$) toward $z = -1$ (at $K = K_\text{max}$).

  \begin{figure}[H]\centering
    \includegraphics[width=0.7\textwidth]{images/pole_map_K.pdf}
    \caption{Closed-loop pole location $z_\text{cl}(K) = 1 - T_s r K$ for varying $K$. Unit circle shown for stability reference.}
    \label{fig:pole_map_K}
  \end{figure}

  \paragraph{Simulated step responses:}
  The discrete-time step response is simulated for $K \in \{20, 40, 80\}$ rad/(s$\cdot$m):
  \begin{equation}
    x[k+1] = (1 - T_s r K) x[k] + T_s r K \cdot x_\text{ref}, \quad x[0] = x_0.
  \end{equation}
  
  \begin{table}[H]\centering
    \begin{tabular}{lccc}
      \toprule
      $K$ [rad/(s$\cdot$m)] & $z_\text{cl}$ & Pole location & Expected behavior \\
      \midrule
      20  & 0.9934 & Close to 1 & Slow convergence, long settling time \\
      40  & 0.9868 & Moderate   & Faster response, good compromise \\
      80  & 0.9736 & Closer to 0 & Fast response, higher control effort \\
      \bottomrule
    \end{tabular}
    \caption{Closed-loop poles for different $K$ values.}
    \label{tab:K_poles}
  \end{table}

  \begin{figure}[H]\centering
    \includegraphics[width=0.8\textwidth]{images/step_K_sweep.pdf}
    \caption{Simulated closed-loop step responses. Larger $K$ yields faster convergence but higher peak velocity commands.}
    \label{fig:step_K_sweep}
  \end{figure}

  \paragraph{Relationship between pole location, $K$, and time response:}
  \begin{itemize}
    \item \textbf{Small $K$} ($z_\text{cl} \approx 1$): Slow exponential decay, long settling time, smooth but sluggish response.
    \item \textbf{Moderate $K$} ($0 < z_\text{cl} < 1$): Faster convergence, monotonic response without oscillation.
    \item \textbf{Large $K$} ($z_\text{cl} < 0$): Sign-alternating (oscillatory) discrete-time response. If $|z_\text{cl}| < 1$, still stable but with overshoot.
    \item \textbf{$K > K_\text{max}$} ($|z_\text{cl}| > 1$): Unstable, diverging oscillations.
  \end{itemize}

  \paragraph{Choice of $K$:}
  The theoretical stability limit is $K_\text{max} \approx 6060$ rad/(s$\cdot$m), but practical constraints are more restrictive:
  \begin{itemize}
    \item \textbf{Actuator saturation}: The motor voltage is limited to $\pm 11$~V. Large $K$ causes large velocity commands for small position errors, potentially saturating the inner velocity loop.
    \item \textbf{Sensor noise}: High $K$ amplifies measurement noise into the control signal.
    \item \textbf{IR sensor range}: The sensor is accurate only for 5--30~cm. Aggressive control may drive the cart outside this range.
  \end{itemize}
  
  \textbf{Selected value:}
  \begin{equation}
    \boxed{K = 40\ \text{rad/(s$\cdot$m)}.}
  \end{equation}
  This gives $z_\text{cl} = 1 - 0.01 \times 0.033 \times 40 = 0.9868$, providing:
  \begin{itemize}
    \item Stable operation well within the stability margin
    \item Reasonable settling time ($\approx 3$--$5$~s for 2\% criterion)
    \item Control effort within actuator limits for typical position steps (0.1--0.2~m)
  \end{itemize}




  \subsection{Design of state estimator gain $L$ using pole placement} % Section 1(d)
  
  A discrete-time Luenberger observer is used to estimate the cart position from the IR sensor measurement:
  \begin{equation}
    \hat{x}[k+1] = A_d \hat{x}[k] + B_d u[k] + L \cdot \nu[k],
  \end{equation}
  where $\nu[k] = y[k] - C\hat{x}[k]$ is the \textbf{innovation} (measurement prediction error).
  
  With $A_d = 1$, $B_d = T_s r$, and $C = -1$:
  \begin{equation}
    \hat{x}[k+1] = \hat{x}[k] + T_s r \cdot u[k] + L \cdot (y[k] + \hat{x}[k]).
  \end{equation}

  \paragraph{Derivation of estimator pole as function of $L$:}
  Define the estimation error:
  \begin{equation}
    e[k] = x[k] - \hat{x}[k].
  \end{equation}
  
  The true state evolves as $x[k+1] = x[k] + T_s r \cdot u[k]$. Subtracting the observer equation:
  \begin{align}
    e[k+1] &= x[k+1] - \hat{x}[k+1] \\
           &= (x[k] + T_s r \cdot u[k]) - (\hat{x}[k] + T_s r \cdot u[k] + L(y[k] - C\hat{x}[k])) \\
           &= (x[k] - \hat{x}[k]) - L(Cx[k] - C\hat{x}[k]) \\
           &= e[k] - L \cdot C \cdot e[k] \\
           &= (1 - LC) \cdot e[k].
  \end{align}
  
  Substituting $C = -1$:
  \begin{equation}
    e[k+1] = (1 - L \cdot (-1)) \cdot e[k] = (1 + L) \cdot e[k].
  \end{equation}
  
  The estimator error dynamics have the closed-loop pole:
  \begin{equation}
    \boxed{z_\text{est}(L) = 1 + L.}
  \end{equation}

  \paragraph{Pole behavior as function of $L$:}
  \begin{itemize}
    \item At $L = 0$: $z_\text{est} = 1$ (no correction, estimator ignores measurements)
    \item As $L$ becomes more negative: pole moves \textbf{left along the real axis}
    \item At $L = -1$: $z_\text{est} = 0$ (deadbeat estimator, instant convergence)
    \item At $L = -2$: $z_\text{est} = -1$ (stability boundary)
  \end{itemize}

  \paragraph{Stability analysis:}
  For the estimator to be stable:
  \begin{equation}
    |z_\text{est}(L)| = |1 + L| < 1.
  \end{equation}
  This yields:
  \begin{equation}
    \boxed{-2 < L < 0.}
  \end{equation}
  
  \textbf{Yes, the estimator can become unstable}:
  \begin{itemize}
    \item If $L > 0$: pole $z_\text{est} > 1$, estimation error grows exponentially.
    \item If $L < -2$: pole $z_\text{est} < -1$, estimation error diverges with oscillations.
  \end{itemize}

  \paragraph{Trade-offs in pole placement:}
  \begin{itemize}
    \item \textbf{Faster estimator} (larger $|L|$, pole closer to 0):
    \begin{itemize}
      \item[+] Faster convergence from wrong initial estimate
      \item[+] Quicker correction of model errors
      \item[$-$] Higher sensitivity to measurement noise (noise is amplified by $L$)
      \item[$-$] Risk of oscillatory behavior if $z_\text{est} < 0$
    \end{itemize}
    \item \textbf{Slower estimator} (smaller $|L|$, pole closer to 1):
    \begin{itemize}
      \item[+] Smoother estimate, better noise rejection
      \item[$-$] Slow convergence from initialization errors
      \item[$-$] Poor tracking of rapid state changes
    \end{itemize}
  \end{itemize}
  
  \textbf{Design guideline}: The estimator pole should typically be 2--6 times faster than the controller pole, ensuring the estimation error decays before significantly affecting control performance.

  \paragraph{Pole placement calculation:}
  With the chosen controller gain $K = 40$ rad/(s$\cdot$m), the controller pole is:
  \begin{equation}
    z_\text{cl} = 1 - T_s r K = 1 - 0.01 \times 0.033 \times 40 = 0.9868.
  \end{equation}
  
  To make the estimator 4 times faster (in continuous-time sense), we use:
  \begin{equation}
    z_\text{est} = z_\text{cl}^4 = 0.9868^4 \approx 0.948.
  \end{equation}
  
  Solving for $L$:
  \begin{equation}
    L = z_\text{est} - 1 = 0.948 - 1 = -0.052.
  \end{equation}

  \paragraph{Selected value:}
  \begin{equation}
    \boxed{L = -0.05 \text{ (dimensionless)}.}
  \end{equation}
  
  This gives $z_\text{est} = 0.95$, which:
  \begin{itemize}
    \item Lies safely within the stability region $-2 < L < 0$
    \item Provides faster convergence than the controller pole
    \item Maintains reasonable noise rejection
  \end{itemize}
  
  \textbf{Note}: The gain $L$ is dimensionless because both $y[k]$ and $x[k]$ have units of meters. The correction term $L \cdot \nu[k]$ has units [m], matching the state.
  
  \paragraph{Alternative gains for experiments:}
  To study the effect of $L$ on estimator performance, we test:
  \begin{table}[H]\centering
    \begin{tabular}{lcc}
      \toprule
      Gain & $z_\text{est}$ & Expected behavior \\
      \midrule
      $L = -0.05$ (slow)  & 0.95 & Slow convergence, smooth estimate \\
      $L = -0.18$ (nominal) & 0.82 & Moderate speed, good compromise \\
      $L = -0.35$ (fast)  & 0.65 & Fast convergence, noisier estimate \\
      \bottomrule
    \end{tabular}
    \caption{Estimator gains tested in experiments.}
    \label{tab:L_gains}
  \end{table}



  \section{Implementation and testing of state estimator and state feedback controller}
  
  Now the designed estimator and controller are implemented on the Arduino, following the assignment structure.
  Data and figures will be populated after running the Arduino experiments and MATLAB post-processing. Each plot placeholder corresponds to a file exported to \texttt{images/}.


  \subsection{Estimator only: wrong initial estimate, different $L$} % Section 2(a)
  The controller is disabled; the estimator starts from a wrong initial position $\hat{x}[0]$ set via \texttt{readValue(5)} in the Arduino code. 
  
  Configuration:
  \begin{itemize}
    \item Estimator active, controller off. The cart is moved manually, or driven with a known, simple input signal.
    \item The state estimator is initialized with a wrong position estimate $\hat{x}[0]$ (e.g. 10$cm$ closer to the wall than reality).
    \item Different values of $L$ are tested (e.g. $L$ = -0.02, -0.07, -0.15).
  \end{itemize}
  For each L:
  \begin{itemize}
    \item The measured distance $y[k]$ (positive, measured by IR sensor) is logged.
    \item The estimated state $\hat{x}[k]$ is logged.
  \end{itemize}

  %%% FIGURE 4: Measured distance to the wall and estimated distance 
  %%% (converted to meters along x, i.e. \hat{y}[k] = - \hat{x}[k]) vs. time for several values of L on one figure, 
  %%% showing how the estimator converges from a wrong initial estimate.

  %\begin{figure}[h!]
  %  \centering
  %  \includegraphics[width=0.48\textwidth]{state_feedback_assignment}}
  %  \caption{Cart measuring its distance to a wall.}
  %  \label{fig:state_feedback_assignment}
  %\end{figure}

  Figure~\ref{fig:estimator_L_sweep} overlays the measured distance $y=-x$ and the estimated distance $-\hat{x}$ for $L\in\{-0.05,-0.18,-0.35\}$. Larger $|L|$ yields faster convergence but more noise on $\hat{x}$ and $\nu$.

  \begin{figure}[H]\centering
    \includegraphics[width=0.8\textwidth]{images/estimator_L_sweep.pdf}
    \caption{Measured vs. estimated distance for different $L$ (wrong initial estimate).}
    \label{fig:estimator_L_sweep}
  \end{figure}

  \paragraph{Interpretation and link to design of state estimator gain $L$:}
  For small $|L|$ (e.g. $L \approx -0.02$, so $z_\text{est} \approx 0.98$):
  \begin{itemize}
    \item The estimator reacts slowly to the discrepancy between measurement and prediction.
    \item The estimate converges slowly to the measurement, a substantial transient remains for a long time.
    \item This matches the formula $z_{est} = 1 + L$: pole close to 1 leads to slow dynamics.
  \end{itemize}

  For moderate $|L|$ (e.g. the designed $L \approx -0.07$, giving $z_{est} \approx 0.93$):
  \begin{itemize}
    \item The estimate converges significantly faster to the measurement.
    \item Noise on the measurement is visible but not overly amplified.
  \end{itemize}

  For large $|L|$ close to -2 (e.g. $L$ = -1.5, so $z_\text{est}$ = -0.5):
  \begin{itemize}
    \item Convergence is very fast, but the estimate closely follows measurement noise, becoming noisy and irregular.
    \item This shows the trade-off: faster convergence vs. noise sensitivity.
  \end{itemize}


  So:
  \begin{itemize}
    \item Convergence speed increases with $|L|$, consistent with $z_\text{est}=1+L$.
    \item Noise on $\hat{x}$ and the innovation $\nu$ grows with $|L|$, illustrating the trade-off from 1(d).
    \item The chosen $L_\text{nom}$ balances settling time ($\approx$~TODO~s) and noise (RMS $\approx$~TODO~m).
  \end{itemize}

  From the plots, one concludes that larger negative $L$ values (closer to −2) give faster convergence of the estimates to the measurements, 
  consistent with the expression $z_\text{est} = 1 + L$ and the trade-offs discussed in Section 1.4 (design of state estimator gain $L$).




  \subsection{Controller only: proportional feedback for different $K$} % Section 2(b)
  The estimator is disabled, feedback uses the raw distance $y[k] = -x[k]$. A 0.15$m$ step reference is applied for $K\in\{K_\text{slow},K_\text{nom},K_\text{fast}\}$. 
  
  Configuration:
  \begin{itemize}
    \item Controller active, estimator off. The controller directly uses the IR measurement (converted to position) for feedback.
    \item The control law becomes: $u[k] = K(r[k]-x[k])$, with $x[k]$ obtained from the IR sensor relation $x[k] = -y[k]$. 
    \item A step in position reference is applied. The car starts at, for example, 30$cm$ from the wall. The reference is stepped from 30$cm$ to 15$cm$ (in x: $x=-0.30$ to $x=-0.15$).
  \end{itemize}
  Repeat this experiment for several values of $K$ (e.g. $K$ = 10,50,100 $rad/(s\cdot m)$).

  %%% FIGURE 5: Position step reference and measured position responses 
  %%% for several K values on a single plot.

  %\begin{figure}[h!]
  %  \centering
  %  \includegraphics[width=0.48\textwidth]{state_feedback_assignment}}
  %  \caption{Cart measuring its distance to a wall.}
  %  \label{fig:state_feedback_assignment}
  %\end{figure}


  %%% FIGURE 6: Corresponding low-level control signals (motor voltages) vs. time 
  %%% for the same K values.

  %\begin{figure}[h!]
  %  \centering
  %  \includegraphics[width=0.48\textwidth]{state_feedback_assignment}}
  %  \caption{Cart measuring its distance to a wall.}
  %  \label{fig:state_feedback_assignment}
  %\end{figure}

  Figure~\ref{fig:controller_K_response} shows position responses and Figure~\ref{fig:controller_K_voltage} the corresponding motor voltages.

  \begin{figure}[H]\centering
    \includegraphics[width=0.8\textwidth]{images/controller_K_response.pdf}
    \caption{Measured position step responses for different $K$ (estimator disabled).}
    \label{fig:controller_K_response}
  \end{figure}

  \begin{figure}[H]\centering
    \includegraphics[width=0.8\textwidth]{images/controller_K_voltage.pdf}
    \caption{Motor voltage commands for the responses in Fig.~\ref{fig:controller_K_response}.}
    \label{fig:controller_K_voltage}
  \end{figure}

  \paragraph{Interpretation and link to design of state feedback controller gain $K$:}
  As shown analytically, the closed-loop pole is
  \begin{equation}
    z_\text{cl}(K) = 1 - T_s r K.
  \end{equation}
  
  For small $K$ (e.g. $K$ = 20 $rad/(s\cdot m)$):
  \begin{itemize}
    \item Pole close to 1: slow rise, long settling time, very smooth control signal.
    \item This matches time-responses with slow approach to the reference.
  \end{itemize}

  For moderate $K$ (e.g. the designed $K$ = 50 $rad/(s\cdot m)$):
  \begin{itemize}
    \item Pole closer to zero: faster response, reduced settling time.
    \item Acceptable overshoot and control signals well within approximately 12$V$.
    \item Good tracking performance.
  \end{itemize}

  For large $K$ close to theoretical maximum $\frac{2}{T_s r}$ (e.g. $K$ = 100 $rad/(s\cdot m)$):
  \begin{itemize}
    \item Pole approaches -1: discrete-time oscillatory response and possibly overshoot.
    \item The control signals reach or exceed saturation (12$V$), so the inner velocity loop cannot follow the demanded angular velocity.
  \end{itemize}


  So: 
  \begin{itemize}
    \item Rise time and steady-state error match the simulated trend from Section 1.3 (design of state feedback controller gain $K$), higher $K$ speeds up the response but increases peak voltage.
    \item For $K=K_\text{fast}$ the voltage briefly saturates, explaining the mild overshoot; this bounds feasible $K$ in practice.
    \item The selected $K_\text{nom}$ avoids saturation while providing $t_s\approx$~TODO~s and negligible steady-state error.
  \end{itemize}

  These observations correspond to the theoretical dependence $z_{cl}(K) = 1 - T_s r K$:
  as $K$ increases, the closed-loop pole moves left towards -1, giving faster dynamics until saturation and discrete oscillations become limiting.

  \paragraph{Choice of $K$:}
  There are both theoretical and practical limits for $K$.
  Theoretical limit: stability in discrete-time requires
  \begin{equation}
    0 < K < \frac{2}{T_s r} \approx 6060\ \text{rad/(s$\cdot$m)}.
  \end{equation}

  Practical limits:
  \begin{itemize}
    \item The low-level motor voltage is bounded: too large $K$ causes saturation, making the actual behavior deviate from the linear model.
    \item Large $K$ amplifies sensor noise and model errors.
    \item The IR sensor has a limited range (approximately 5-30$cm$). Overshoot may drive the cart outside this range, resulting in invalid measurements.
  \end{itemize}
  In light of these constraints, a value around
  \begin{equation}
    K = 40\text{rad/(s\cdot m)}
  \end{equation}
  offers a good compromise between tracking performance and actuator/sensor limitations, 
  and is therefore used in the remainder of the assignment.

  




  \subsection{Combined estimator and controller: influence of estimator pole choice} % Section 2(c)
  
  In this experiment, both the state feedback controller and the state estimator are active. The controller uses the estimated position $\hat{x}[k]$ instead of the true position:
  \begin{equation}
    u[k] = K(x_\text{ref}[k] - \hat{x}[k]).
  \end{equation}
  
  Per the assignment specification, the estimator is designed to be \textbf{10 times slower} than the controller.

  \paragraph{Recap of closed-loop poles:}
  From Sections 1(c) and 1(d), the closed-loop poles are:
  \begin{itemize}
    \item \textbf{Controller pole}: $z_\text{cl}(K) = 1 - T_s r K$
    \item \textbf{Estimator pole}: $z_\text{est}(L) = 1 + L$
  \end{itemize}
  
  With $K = 40$ rad/(s$\cdot$m):
  \begin{equation}
    z_\text{cl} = 1 - 0.01 \times 0.033 \times 40 = 0.9868.
  \end{equation}

  \paragraph{Designing the estimator to be 10$\times$ slower:}
  A pole closer to 1 corresponds to slower dynamics. To make the estimator 10 times slower in continuous-time:
  \begin{enumerate}
    \item Convert controller pole to continuous-time: $s_\text{cl} = \frac{\ln(z_\text{cl})}{T_s} = \frac{\ln(0.9868)}{0.01} \approx -1.33$ rad/s
    \item Estimator pole 10$\times$ slower: $s_\text{est} = \frac{s_\text{cl}}{10} = -0.133$ rad/s
    \item Convert back to discrete-time: $z_\text{est} = e^{s_\text{est} \cdot T_s} = e^{-0.00133} \approx 0.9987$
  \end{enumerate}
  
  Solving for $L$ using $z_\text{est} = 1 + L$:
  \begin{equation}
    L = z_\text{est} - 1 = 0.9987 - 1 = -0.0013.
  \end{equation}
  
  \textbf{Selected slow estimator gain}:
  \begin{equation}
    \boxed{L_\text{slow} = -0.00132 \text{ (dimensionless)}.}
  \end{equation}
  This matches the value used in the Arduino code for Section 2(c).

  \paragraph{Full closed-loop system poles:}
  The combined system (controller + estimator) has two poles:
  \begin{equation}
    \boxed{\lambda_1 = z_\text{cl} = 0.9868, \qquad \lambda_2 = z_\text{est} = 0.9987.}
  \end{equation}
  
  \textbf{Key observation (Separation Principle)}: The controller pole $z_\text{cl}$ depends only on $K$, and the estimator pole $z_\text{est}$ depends only on $L$. The two dynamics are decoupled---the estimator design does not affect the controller pole, and vice versa.

  \paragraph{Experiments:}
  Two experiments are performed with the slow estimator ($L = -0.00132$):
  \begin{enumerate}
    \item \textbf{Good initial estimate}: $\hat{x}[0] \approx x[0]$. A step reference is applied.
    \item \textbf{Wrong initial estimate}: $\hat{x}[0] = x[0] + \Delta x$ (e.g., 10--15~cm offset). A step reference is applied simultaneously.
  \end{enumerate}

  \begin{figure}[H]\centering
    \includegraphics[width=0.8\textwidth]{images/est_ctrl_good.pdf}
    \caption{Combined estimator and controller with good initial estimate. The estimator tracks the measurement, and the controller achieves the design settling time.}
    \label{fig:est_ctrl_good}
  \end{figure}

  \begin{figure}[H]\centering
    \includegraphics[width=0.8\textwidth]{images/est_ctrl_bad.pdf}
    \caption{Combined estimator and controller with wrong initial estimate. The slow estimator takes a long time to converge, but the control performance remains acceptable.}
    \label{fig:est_ctrl_bad}
  \end{figure}

  \paragraph{Analysis of results:}
  
  \textbf{1. Good initial estimate} ($\hat{x}[0] \approx x[0]$):
  \begin{itemize}
    \item The estimation error is small from the start.
    \item The controller responds to the step reference with dynamics governed by $z_\text{cl} = 0.9868$.
    \item Control performance is satisfactory---the cart reaches the reference position with the expected settling time.
  \end{itemize}
  
  \textbf{2. Wrong initial estimate} ($\hat{x}[0] \neq x[0]$):
  \begin{itemize}
    \item The estimator starts with a large error and converges slowly (pole at 0.9987).
    \item During convergence, the controller acts on a biased estimate, causing suboptimal transient behavior.
    \item However, the \textbf{steady-state} control performance is unaffected---once the estimator converges, the controller achieves zero steady-state error.
  \end{itemize}
  
  \paragraph{Is control performance satisfactory?}
  \begin{itemize}
    \item \textbf{Good initialization}: Yes, the controller meets design specifications.
    \item \textbf{Wrong initialization}: Transient performance is degraded due to the slow estimator, but steady-state is correct. For better transient performance, a faster estimator (larger $|L|$) should be used.
  \end{itemize}
  
  \paragraph{Recommended estimator pole placement:}
  For practical applications, the estimator should be \textbf{faster} than the controller (2--6 times), not slower. This ensures:
  \begin{itemize}
    \item Quick convergence from initialization errors
    \item Minimal impact of estimation error on control performance
  \end{itemize}
  A value such as $L = -0.05$ to $-0.18$ (giving $z_\text{est} = 0.82$--$0.95$) would be more appropriate for real applications.

  \paragraph{Separation principle verification:}
  The closed-loop transfer function from reference to output can be written as:
  \begin{equation}
    \frac{X(z)}{X_\text{ref}(z)} = \frac{T_s r K}{(z - z_\text{cl})(z - z_\text{est})} \cdot (z - z_\text{est}) = \frac{T_s r K}{z - z_\text{cl}}.
  \end{equation}
  The estimator pole $z_\text{est}$ cancels in the transfer function from reference to output, confirming that:
  \begin{itemize}
    \item The \textbf{reference tracking} dynamics depend only on $z_\text{cl}$ (controller pole).
    \item The estimator pole affects only the \textbf{estimation error} dynamics, not the control dynamics.
    \item This is the \textbf{separation principle}: controller and estimator can be designed independently.
  \end{itemize}

  \newpage
  \bibliographystyle{plain}
  \bibliography{references}

\end{document}
