\documentclass{article}
\usepackage{graphicx}
\usepackage{adjustbox}
\usepackage{natbib}
\usepackage{hyperref}
\usepackage{xurl}
\usepackage{geometry}
\usepackage{amsmath}
\usepackage{booktabs}
\usepackage{fancyhdr}
\usepackage{setspace}
\usepackage{listings}
\usepackage{xcolor}
\usepackage{float}
\usepackage{subfig}
\geometry{margin=1in}

% Define MATLAB style for listings
\lstdefinestyle{MatlabStyle}{
  language=Matlab,
  basicstyle=\ttfamily\footnotesize,
  keywordstyle=\color{blue},
  stringstyle=\color{red},
  commentstyle=\color{green!50!black},
  numbers=left,
  numberstyle=\tiny\color{gray},
  stepnumber=1,
  numbersep=5pt,
  backgroundcolor=\color{white},
  showspaces=false,
  showstringspaces=false,
  showtabs=false,
  frame=single,
  tabsize=2,
  captionpos=b,
  breaklines=true,
  breakatwhitespace=false,
  escapeinside={\%*}{*)}
}

\begin{document}

  \begin{titlepage}
    \thispagestyle{empty}
    \pagenumbering{Roman}
    \frenchspacing

    \noindent
    \begin{tabular}{@{} c @{\hspace{0.1cm}} c @{\hspace{0.2cm}} c @{}}
      \adjustbox{valign=c}{%
        \includegraphics[height=2cm, trim=0 0 0 0, clip]{sedes.pdf}%
      } &
      \adjustbox{valign=c}{%
        \rule{0.5pt}{2cm}%
      } &
      \adjustbox{valign=c}{%
        \includegraphics[height=2cm, trim=0 0 0 0, clip]{logoFirW.jpg}%
      }
    \end{tabular}
    \hfill

    \vspace*{2cm}

    \begin{center}
      \begin{minipage}[t]{\textwidth}
        \begin{center}
          \large\textbf{B-KUL-H04X3A: Control Theory}
        \end{center}
        \vspace{1cm}
        \begin{center}
          \textbf{Team members:}\\[2mm]
          Lefebure Tiebert (r0887630)\\
          Campaert Lukas (r0885501)\\
        \end{center}
        \vspace{1cm}
        \begin{center}
          \Huge\textbf{Assignment 3: State Feedback and State Estimation}
        \end{center}
        \vspace{1cm}
        \begin{center}
          \underline{Professor:}\\[2mm]
          Prof. Dr. Ir. Jan Swevers\\
        \end{center}
        \vspace{2cm}
        \makebox[\textwidth]{Academic Year 2025-2026}
      \end{minipage}
    \end{center}
  \end{titlepage}

  \newpage
  \vspace*{3.2cm}\vfill

  \begin{center}
    \Large\textbf{\textit{Declaration of Originality}}
  \end{center}

  \noindent \textit{We hereby declare that this submitted draft is entirely our own, subject to feedback and support given us by the didactic team, and subject to lawful cooperation which was agreed with the same didactic team. Regarding this draft, we also declare that:}

  \begin{enumerate}
    \item \textit{Note has been taken of the text on academic integrity \url{https://eng.kuleuven.be/studeren/masterproef-en-papers/documenten/20161221-academischeintegriteit-okt2016.pdf}.}
    \item \textit{No plagiarism has been committed as described on \url{https://eng.kuleuven.be/studeren/masterproef-en-papers/plagiaat\#Definitie:\%20wat\%20is\%20plagiaat?}.}
    \item \textit{All experiments, tests, measurements, \ldots, have been performed as described in this draft, and no data or measurement results have been manipulated.}
    \item \textit{All sources employed in this draft --- including internet sources --- have been correctly referenced.}
  \end{enumerate}

  \newpage

  \section{State estimator and state feedback controller design using pole placement}

  \paragraph{Choice of state and input:}
  Unless stated otherwise, the sampling time is $T_s$ = 0.01$s$ and the wheel radius is $r$ = $D/2$ = 0.033$m$.
  The goal of this assignment is to control the position of the cart along a straight line.
  The state is the cart position $x(t)$ [$m$] along the $x$-axis, measured (with an infrared sensor) from the wall at the origin $x$=0 ($x$<0 when the cart is in front of the wall). 
  The input $u(t)=\omega(t)$ [$rad/s$] is the common wheel velocity command to be applied to both motors.

  \subsection{Discrete-time state equation} % Section 1(a)
  
  \paragraph{Continuous-time state equation:} 
  The velocity control loop from Assignment 2 is assumed to be ideal, so the low-level controller is considered to track the velocity setpoint perfectly.
  The continuous-time state equation is then simply the kinematic relation $v(t) = \dot{x}(t) = r \cdot \omega(t) = r \cdot u(t)$ [$m/s$], where $\omega(t)$ is the average wheel speed. 
  In state-space form with state $x(t)$ and input $u(t)=\omega(t)$:
  \begin{equation}
    v(t) = \dot{x}(t) = A_c x(t) + B_c u(t), \quad A_c = 0,\quad B_c = r.
  \end{equation}

  \paragraph{Forward Euler discretization:} 
  Using $x[k+1] = x[k] + T_s \cdot \dot{x}[k]$,
  the continuous-time equation can be written in discrete-time state-space form as:
  \begin{equation}
    x[k+1] = x[k] + T_s r u[k] = A_d x[k] + B_d u[k],
  \end{equation}
  with
  \begin{equation}
    A_d = [1], \qquad B_d = [T_s r] = [0.01\text{s} \cdot 0.033\text{m}] = [3.3\times10^{-4}\ \text{m \cdot s}].
  \end{equation}
  For the estimator, the averaging $u[k] \approx (\omega_A[k]+\omega_B[k])/2$ is also used to incorporate measured wheel speeds.


  \subsection{Measurement equation} % Section 1(b)

  The front infrared (IR) sensor measures the distance to the wall, which is a positive quantity, i.e. $y[k] = -x[k] + v_d[k]$ with $v_d$ the measurement noise. 
  The discrete-time state-space model is:
  \begin{equation}
    y[k] = C x[k] + D u[k],\qquad C = [-1],\qquad D = [0].
  \end{equation}
  The positive distance reading corresponds to $-x[k]$, consistent with Fig.~\ref{fig:IRsensor}.
  The input $u[k]$ does not directly appear in the sensor output $y[k]$, meaning $D=[0]$.

  %%% FIGURE 1: Schematic of cart and coordinates: wall at x = 0, 
  %%% cart on negative x, infrared sensor in front measuring distance y = -x.

  %\begin{figure}[h!]
  %  \centering
  %  \includegraphics[width=0.48\textwidth]{state_feedback_assignment}}
  %  \caption{Cart measuring its distance to a wall.}
  %  \label{fig:state_feedback_assignment}
  %\end{figure}



  \subsection{Design of state feedback controller gain $K$ using pole placement} % Section 1(c)
  The position controller acts on the measured position (or state estimate) and outputs the desired wheel angular velocity. 
  Assuming full state feedback (no estimator yet), the controller is taken as a proportional feedback on position error:
  \begin{itemize}
    \item Reference position: $r[k]$ [$m$], defined along the same $x$-axis.
    \item Position error: $e[k]$ = $r[k]$ - $x[k]$.
    \item Control law: $u[k] = K e[k] = K (r[k] - x[k])$, where K has units [$\frac{rad/s}{m}$].
  \end{itemize}
  

  \paragraph{Closed-loop state equation and pole as function of $K$:}
  Substitute the control law into the discrete-time state equation:
  \begin{equation}
    x[k+1] = A_d x[k] + B_d u[k] = A_d x[k] + B_d K (r[k] - x[k])
  \end{equation}
  with $A_d = 1$ and $B_d = T_s r$,
  \begin{equation}
    x[k+1] = (1 - T_s r K)x[k] + T_s r K r[k].
  \end{equation}

  For a constant step reference $r[k] = r_0$, the error dynamics are:
  \begin{equation}
    e[k] = r_0 - x[k],
    \qquad
    e[k+1] = r_0 - x[k+1] = r_0 - [(1 - T_s r K)x[k] + T_s r K r_0].
  \end{equation}
  Using $x[k] = r_0 - e[k]$:
  \begin{equation}
    e[k+1] = r_0 - (1 - T_s r K)(r_0 - e[k]) - T_s r K r_0 = r_0 - (1 - T_s r K)r_0 + (1 - T_s r K)e[k] - T_s r.
  \end{equation}
  Hence the error dynamics are:
  \begin{equation}
    e[k+1] = (1 - T_s r K)e[k].
  \end{equation}
  So the discrete-time closed-loop pole is:
  \begin{equation}
    z_\text{cl}(K) = 1 - T_s r K.
  \end{equation}

  \begin{itemize}
    \item When $K$ increases, the pole moves left along the real axis.
    \item At $K$=0, $z_\text{cl} = 1$ (open-loop integrator).
    \item For increasing $K$, the pole crosses zero and becomes negative.
  \end{itemize}


  \paragraph{Stability as a function of $K$:}
  For discrete-time stability:
  \begin{equation}
    |z_\text{cl}(K)| = |1 - T_s r K| < 1 
  \end{equation}
  That gives:
  \begin{equation}
    0 < K < \frac{2}{T_s r} \approx 6.1\times10^3 \text{$rad/(s\cdotm)$}.
  \end{equation}

  \begin{itemize}
    \item For 0 < $K$ < $\frac{2}{T_s r}$, the closed-loop system is stable.
    \item For $K$ > $\frac{2}{T_s r}$, the discrete-time pole lies outside the unit circle and the closed-loop system becomes unstable.
    \item A faster response is obtained by choosing $z_\text{cl}$ closer to 0, at the cost of higher velocity commands.
  \end{itemize}

  %%% FIGURE 2: Pole location z_cl(K) = 1 - T_s r K as function of K.
  %%% Straight line on real axis from z = 1 at K = 0 to z = -1 at K = 2/(T_s r).

  %\begin{figure}[h!]
  %  \centering
  %  \includegraphics[width=0.48\textwidth]{state_feedback_assignment}}
  %  \caption{Cart measuring its distance to a wall.}
  %  \label{fig:state_feedback_assignment}
  %\end{figure}

  \paragraph{Pole-zero map.} Figure~\ref{fig:pole_map_K} shows $z_\text{cl}(K)$ for $K\in\{K_\text{slow},K_\text{nom},K_\text{fast}\}=\{80,120,250\}$~rad/(s$\cdot$m). %TODO replace values if retuned.

  \begin{figure}[H]\centering
    \fbox{\parbox{0.8\textwidth}{Placeholder for pole-zero map from MATLAB (images/pole\_map\_K.pdf).}}
    \caption{Closed-loop pole location versus $K$.}
    \label{fig:pole_map_K}
  \end{figure}



  \paragraph{Closed-loop step responses for varying $K$:}
  Simulate the discrete-time system for a unit step in position reference:
  \begin{equation}
    x[k+1] = (1 - T_s r K)x[k] + T_s r K r_0, \quad x[0] = x_0
  \end{equation}

  Typical behaviors:
  \begin{itemize}
    \item Small $K$ (e.g. $K$ ≈ 0): pole $z_\text{cl}$ ≈ 1, very slow response, large settling time.
    \item Moderate $K$: pole $z_\text{cl}$ moves closer to the origin, faster convergence, reasonably small rise and settling times, no oscillation because the pole remains real and between -1 and 1 (inside the unit circle).
    \item Large $K$ (near upper stability limit, $K$ ≈ $\frac{2}{T_s r}$): pole $z_\text{cl}$ approaches -1, oscillatory (sign-alternating) behavior in discrete-time with long settling and potential overshoot.
  \end{itemize}

  %%% FIGURE 3: Simulated step responses x[k] for several K values on one plot
  %%% (e.g. K = 10,50,100,200 rad/(s*m)), showing increasing speed and changes in overshoot.

  %\begin{figure}[h!]
  %  \centering
  %  \includegraphics[width=0.48\textwidth]{state_feedback_assignment}}
  %  \caption{Cart measuring its distance to a wall.}
  %  \label{fig:state_feedback_assignment}
  %\end{figure}

  \paragraph{Step responses.} Figure~\ref{fig:step_K_sweep} groups simulated position step responses $x[k]$ for the same $K$ sweep. 
  Larger $K$ yields shorter rise time and higher control action; excessive $K$ would saturate the motors and could destabilize the discretized model.

  \begin{figure}[H]\centering
    \fbox{\parbox{0.8\textwidth}{Placeholder for simulated step responses (images/step\_K\_sweep.pdf).}}
    \caption{Simulated closed-loop step responses for varying $K$.}
    \label{fig:step_K_sweep}
  \end{figure}


  \paragraph{Choice of $K$ and justification:}
  From the pole expression:
  \begin{equation}
    z_\text{cl}(K) = 1 - T_s r K,
  \end{equation}
  and the stability interval 0 < $K$ < $\frac{2}{T_s r}$, a practical design choice must also respect:
  \begin{itemize}
    \item Actuator voltage limits: inner velocity loop saturates around 12$V$,
    \item Sensor noise: too high $K$ demands very large velocities for small position errors,
    \item Desired settling time: fast enough, but without overshoot or oscillations.
  \end{itemize}

  For example, with typical hardware values (sampling time $T_s$ = 0.01$s$, wheel radius $r$ = 0.033$m$), the upper theoretical bound is
  \begin{equation}
    K_\text{max} = \frac{2}{T_s r} = \frac{2}{0.01\times0.033} \approx 6060 \text{rad/(s$\cdot$m)}.
  \end{equation}

  Experiments (step responses and measured voltages) typically show that:
  \begin{itemize}
    \item Very small $K$ (e.g. $K$ = 10 $rad/(s\cdot m)$) gives a very slow response.
    \item Large $K$ (e.g. $K$ ≈ 100 $rad/(s\cdot m)$) leads to faster response but can demand voltages close to or beyond 12$V$. 
    \item An intermediate value such as $K$ = 50 $rad/(s\cdot m)$ often gives a good compromise: fast response, acceptable overshoot, and actuator voltage below the saturation limit.
  \end{itemize}

  So the chosen value is, for example:
  \begin{equation}
    K = 40 \text{rad/(s$\cdot$m)}.
  \end{equation}
  This lies well within the theoretical stability range and respects practical voltage constraints.


  \paragraph{Gain selection.} A nominal choice $K_\text{nom}=120$~rad/(s$\cdot$m) gives $z_\text{cl}\approx0.96$, keeping the velocity within $\pm 20$~rad/s for 0.2~m position steps while preserving stability margin. This value will be updated after on-cart tests; the MATLAB script exports the active value to the Arduino code.




  \subsection{Design of state estimator gain $L$ using pole placement} % Section 1(d)
  A discrete-time Luenberger observer is used:
  \begin{equation}
    \hat{x}[k+1]=A_d\hat{x}[k]+B_d u[k]+L(y[k]-C\hat{x}[k]),
  \end{equation}
  with $A_d=1$, $B_d= T_s r$, $C=-1$.
  Define the estimation error as:
  \begin{equation}
    e_x[k]=x[k]-\hat{x}[k].
  \end{equation}
  Using the discrete-time state equation $x[k+1] = A_d x[k] + B_d u[k]$, the error dynamics are:
  \begin{equation}
    e_x[k+1] = x[k+1] - \hat{x}[k+1] = (A_d x[k] + B_d u[k]) - (A_d \hat{x}[k] + B_d u[k] + L(y[k]-C\hat{x}[k])).
  \end{equation}
  Substitute $y[k] = C x[k]$:
  \begin{equation}
    e_x[k+1] = A_d(x[k]-\hat{x}[k]) - L(Cx[k]-C\hat{x}[k]) = (A_d - L C)e_x[k].
  \end{equation}
  With $A_d=1$ and $C=-1$:
  \begin{equation}
    A_d - L C = 1 + L,
  \end{equation}

  meaning the error dynamics become:
  \begin{equation}
    e_x[k+1] = (1 + L)e_x[k].
  \end{equation}
  So the estimator discrete-time closed-loop pole is:
  \begin{equation}
    z_\text{est}(L) = 1 + L.
  \end{equation}


  \paragraph{Stability as a function of L:}
  For discrete-time stability:
  \begin{equation}
    |z_\text{est}(L)| = |1 + L| < 1.
  \end{equation}
  This yields:
  \begin{equation}
    -2 < L < 0.
  \end{equation}

  So for stability:
  \begin{itemize}
    \item $L$ must be negative.
    \item As $L \rightarrow 0^-$, $z_\text{est} \rightarrow 1$: estimator becomes very slow.
    \item As $L \rightarrow -2^+$, $z_\text{est} \rightarrow -1$: estimator becomes fast but oscillatory and highly sensitive to noise.
  \end{itemize}
  
  Because $C=-1$, negative $L$ moves the estimator pole inside the unit circle, while $L>0$ makes it unstable. 
  Fast convergence requires $|z_\text{est}|$ small, but a large $|L|$ amplifies sensor noise.

  \paragraph{Trade-offs in choosing $L$:}
  With a faster estimator ($L$ more negative):
  \begin{itemize}
    \item pole closer to 0 or -1, faster convergence of $\hat{x}$ to $x$, but...
    \item stronger amplification of measurement noise, potentially leading to a "nervous" estimate and noisy control action.
  \end{itemize}
  With a slower estimator ($L$ closer to 0):
  \begin{itemize}
    \item smoother estimate, less sensitive to noise, but..
    \item slow convergence from wrong initial conditions, and slower correction of model errors.
  \end{itemize}
  Typical guideline: estimator poles should be 2-6 times faster than controller poles, so that the estimation error decays quickly without dominating the closed-loop dynamics.
  Let the chosen controller pole be 
  \begin{equation}
    z_\text{cl} = 1 - T_s r K.
  \end{equation}
  To get an estimator pole 4x faster in continuous-time, approximate:
  \begin{equation}
    p_c = \frac{ln(z_{cl})}{T_s}, \quad p_e = 4 p_c, \quad z_{est} = e^{T_s p_e} = z_{cl}^{4}.
  \end{equation}
  Then choose $L$ such that
  \begin{equation}
    z_{est} = 1 + L \Rightarrow L = z_{est} - 1 = z_{cl}^{4} - 1.
  \end{equation}
  For example, with $T_s$ = 0.01$s$, $r$ = 0.033$m$, and $K$ = 50$rad/(s\cdot m)$, we get
  \begin{equation}
    z_{cl} = 1 - T_s r K = 1 - 0.01 \cdot 0.033 \cdot 50 = 0.9835,
  \end{equation}
  so
  \begin{equation}
    z_{est} = z_{cl}^{4} = 0.9835^{4} \approx 0.934,
  \end{equation}
  and
  \begin{equation}
    L = z_{est} - 1 ≈ 0.934 - 1 = -0.066.
  \end{equation}
  So a representative design choice is:
  \begin{equation}
    L ≈ -0.07,
  \end{equation}
  dimensionless, because $y[k]$ and $x[k]$ both have units of meters [$m$].
  This value lies in the range $-2<L<0$, gives a faster pole than the controller, and forms a reasonable compromise between speed and noise sensitivity.


  \paragraph{Pole placement and trade-offs.} We select $L$ such that the estimator is $3$--$5$ times faster than the controller for tasks in Section~2(a)--(b), then set it ten times slower than the controller for Section~2(c) per the specification. The practical limits are set by sensor noise and discretization: $|1+L|<1\Rightarrow -2<L<0$.
  \paragraph{Chosen gain.} A nominal value $L_\text{nom}=-0.18$ gives $p_\text{est}=0.82$ (fast but noise-aware). Alternative gains $L\in\{-0.05,-0.18,-0.35\}$ are swept in the experiments below.



  \section{Implementation and testing of state estimator and state feedback controller}
  
  Now the designed estimator and controller are implemented on the Arduino, following the assignment structure.
  Data and figures will be populated after running the Arduino experiments and MATLAB post-processing. Each plot placeholder corresponds to a file exported to \texttt{images/}.


  \subsection{Estimator only: wrong initial estimate, different $L$} % Section 2(a)
  The controller is disabled; the estimator starts from a wrong initial position $\hat{x}[0]$ set via \texttt{readValue(5)} in the Arduino code. 
  
  Configuration:
  \begin{itemize}
    \item Estimator active, controller off. The cart is moved manually, or driven with a known, simple input signal.
    \item The state estimator is initialized with a wrong position estimate $\hat{x}[0]$ (e.g. 10$cm$ closer to the wall than reality).
    \item Different values of $L$ are tested (e.g. $L$ = -0.02, -0.07, -0.15).
  \end{itemize}
  For each L:
  \begin{itemize}
    \item The measured distance $y[k]$ (positive, measured by IR sensor) is logged.
    \item The estimated state $\hat{x}[k]$ is logged.
  \end{itemize}

  %%% FIGURE 4: Measured distance to the wall and estimated distance 
  %%% (converted to meters along x, i.e. \hat{y}[k] = - \hat{x}[k]) vs. time for several values of L on one figure, 
  %%% showing how the estimator converges from a wrong initial estimate.

  %\begin{figure}[h!]
  %  \centering
  %  \includegraphics[width=0.48\textwidth]{state_feedback_assignment}}
  %  \caption{Cart measuring its distance to a wall.}
  %  \label{fig:state_feedback_assignment}
  %\end{figure}

  Figure~\ref{fig:estimator_L_sweep} overlays the measured distance $y=-x$ and the estimated distance $-\hat{x}$ for $L\in\{-0.05,-0.18,-0.35\}$. Larger $|L|$ yields faster convergence but more noise on $\hat{x}$ and $\nu$.

  \begin{figure}[H]\centering
    \fbox{\parbox{0.8\textwidth}{Placeholder for estimator convergence plots (images/estimator\_L\_sweep.pdf).}}
    \caption{Measured vs. estimated distance for different $L$ (wrong initial estimate).}
    \label{fig:estimator_L_sweep}
  \end{figure}

  \paragraph{Interpretation and link to design of state estimator gain $L$:}
  For small $|L|$ (e.g. $L$ ≈ -0.02, so $z_\text{est}$ ≈ 0.98):
  \begin{itemize}
    \item The estimator reacts slowly to the discrepancy between measurement and prediction.
    \item The estimate converges slowly to the measurement, a substantial transient remains for a long time.
    \item This matches the formula $z_{est} = 1 + L$: pole close to 1 leads to slow dynamics.
  \end{itemize}

  For moderate $|L|$ (e.g. the designed $L$ ≈ -0.07, giving $z_{est}$ ≈ 0.93):
  \begin{itemize}
    \item The estimate converges significantly faster to the measurement.
    \item Noise on the measurement is visible but not overly amplified.
  \end{itemize}

  For large $|L|$ close to -2 (e.g. $L$ = -1.5, so $z_\text{est}$ = -0.5):
  \begin{itemize}
    \item Convergence is very fast, but the estimate closely follows measurement noise, becoming noisy and irregular.
    \item This shows the trade-off: faster convergence vs. noise sensitivity.
  \end{itemize}


  So:
  \begin{itemize}
    \item Convergence speed increases with $|L|$, consistent with $z_\text{est}=1+L$.
    \item Noise on $\hat{x}$ and the innovation $\nu$ grows with $|L|$, illustrating the trade-off from 1(d).
    \item The chosen $L_\text{nom}$ balances settling time ($\approx$~TODO~s) and noise (RMS $\approx$~TODO~m).
  \end{itemize}

  From the plots, one concludes that larger negative $L$ values (closer to −2) give faster convergence of the estimates to the measurements, 
  consistent with the expression $z_\text{est} = 1 + L$ and the trade-offs discussed in Section 1.4 (design of state estimator gain $L$).




  \subsection{Controller only: proportional feedback for different $K$} % Section 2(b)
  The estimator is disabled, feedback uses the raw distance $y[k] = -x[k]$. A 0.15$m$ step reference is applied for $K\in\{K_\text{slow},K_\text{nom},K_\text{fast}\}$. 
  
  Configuration:
  \begin{itemize}
    \item Controller active, estimator off. The controller directly uses the IR measurement (converted to position) for feedback.
    \item The control law becomes: $u[k] = K(r[k]-x[k])$, with $x[k]$ obtained from the IR sensor relation $x[k] = -y[k]$. 
    \item A step in position reference is applied. The car starts at, for example, 30$cm$ from the wall. The reference is stepped from 30$cm$ to 15$cm$ (in x: $x=-0.30$ to $x=-0.15$).
  \end{itemize}
  Repeat this experiment for several values of $K$ (e.g. $K$ = 10,50,100 $rad/(s\cdot m)$).

  %%% FIGURE 5: Position step reference and measured position responses 
  %%% for several K values on a single plot.

  %\begin{figure}[h!]
  %  \centering
  %  \includegraphics[width=0.48\textwidth]{state_feedback_assignment}}
  %  \caption{Cart measuring its distance to a wall.}
  %  \label{fig:state_feedback_assignment}
  %\end{figure}


  %%% FIGURE 6: Corresponding low-level control signals (motor voltages) vs. time 
  %%% for the same K values.

  %\begin{figure}[h!]
  %  \centering
  %  \includegraphics[width=0.48\textwidth]{state_feedback_assignment}}
  %  \caption{Cart measuring its distance to a wall.}
  %  \label{fig:state_feedback_assignment}
  %\end{figure}

  Figure~\ref{fig:controller_K_response} shows position responses and Figure~\ref{fig:controller_K_voltage} the corresponding motor voltages.

  \begin{figure}[H]\centering
    \fbox{\parbox{0.8\textwidth}{Placeholder for measured step responses without estimator (images/controller\_K\_response.pdf).}}
    \caption{Measured position step responses for different $K$ (estimator disabled).}
    \label{fig:controller_K_response}
  \end{figure}

  \begin{figure}[H]\centering
    \fbox{\parbox{0.8\textwidth}{Placeholder for control signals (images/controller\_K\_voltage.pdf).}}
    \caption{Motor voltage commands for the responses in Fig.~\ref{fig:controller_K_response}.}
    \label{fig:controller_K_voltage}
  \end{figure}

  \paragraph{Interpretation and link to design of state feedback controller gain $K$:}
  As shown analytically, the closed-loop pole is
  \begin{equation}
    \z{cl}(K) = 1 - T_s r K.
  \end{equation}
  
  For small $K$ (e.g. $K$ = 20 $rad/(s\cdot m)$):
  \begin{itemize}
    \item Pole close to 1: slow rise, long settling time, very smooth control signal.
    \item This matches time-responses with slow approach to the reference.
  \end{itemize}

  For moderate $K$ (e.g. the designed $K$ = 50 $rad/(s\cdot m)$):
  \begin{itemize}
    \item Pole closer to zero: faster response, reduced settling time.
    \item Acceptable overshoot and control signals well within ≈ 12$V$.
    \item Good tracking performance.
  \end{itemize}

  For large $K$ close to theoretical maximum $\frac{2}{T_s r}$ (e.g. $K$ = 100 $rad/(s\cdot m)$):
  \begin{itemize}
    \item Pole approaches -1: discrete-time oscillatory response and possibly overshoot.
    \item The control signals reach or exceed saturation (12$V$), so the inner velocity loop cannot follow the demanded angular velocity.
  \end{itemize}


  So: 
  \begin{itemize}
    \item Rise time and steady-state error match the simulated trend from Section 1.3 (design of state feedback controller gain $K$), higher $K$ speeds up the response but increases peak voltage.
    \item For $K=K_\text{fast}$ the voltage briefly saturates, explaining the mild overshoot; this bounds feasible $K$ in practice.
    \item The selected $K_\text{nom}$ avoids saturation while providing $t_s\approx$~TODO~s and negligible steady-state error.
  \end{itemize}

  These observations correspond to the theoretical dependence $z_{cl}(K) = 1 - T_s r K$:
  as $K$ increases, the closed-loop pole moves left towards -1, giving faster dynamics until saturation and discrete oscillations become limiting.

  \paragraph{Choice of $K$:}
  There are both theoretical and practical limits for $K$.
  Theoretical limit: stability in discrete-time requires
  \begin{equation}
    0 < K < \frac{2}{T_s}.
  \end{equation}

  Practical limits:
  \begin{itemize}
    \item The low-level motor voltage is bounded: too large $K$ causes saturation, making the actual behavior deviate from the linear model.
    \item Large $K$ amplifies sensor noise and model errors.
    \item The IR sensor has a limited range (approximately 5-30$cm$). Overshoot may drive the cart outside this range, resulting in invalid measurements.
  \end{itemize}
  In light of these constraints, a value around
  \begin{equation}
    K = 40\text{rad/(s\cdot m)}
  \end{equation}
  offers a good compromise between tracking performance and actuator/sensor limitations, 
  and is therefore used in the remainder of the assignment.

  




  \subsection{Combined estimator and controller: influence of estimator pole choice} % Section 2(c)
  In this final experiment, both the state feedback controller and the state estimator are active. 
  The controller no longer uses the true position $x[k]$, but the estimated position $\hat{x}[k]$: 
  \begin{equation}
    u[k] = K(x_{ref}[k] - \hat{x}[k]).
  \end{equation}
  Because the measurement equation is
  \begin{equation}
    y[k] = -x[k], \quad C = -1
  \end{equation}
  the controller receives no direct measurement of $x[k]$, all feedback passes through the estimator.

  \paragraph{Controller closed-loop pole $z_{cl}$:}
  From the discretized model
  \begin{equation}
    x[k+1] = x[k] + T_s r u[k]
  \end{equation}
  and the control law $u[k] = K(x_{ref}[k] - \hat{x}[k])$, the closed-loop state update becomes:
  \begin{equation}
    x[k+1] = (1 - T_s r K)x[k] + T_s r K x_{ref}[k].
  \end{equation}
  Hence the controller pole is:
  \begin{equation}
    z_{cl}(K) = 1 - T_s r K = 1 - 0.01 \cdot 0.033 \cdot 40 = 0.9967.
  \end{equation}
  This pole does not depend on the estimator.

  \paragraph{Designing the estimator to be 10x slower:}
  The observer is:
  \begin{equation}
    \hat{x}[k+1] = \hat{x}[k] + T_s r u[k] + L(y[k] - hat{y}[k]).
  \end{equation}
  Given $y[k] = -x[k]$, and $\hat{y}[k] = -\hat{x}[k]$, the innovation becomes:
  \begin{equation}
    y[k] - \hat{y}[k] = -x[k] + \hat{x}[k] = -e[k].
  \end{equation}
  So the observer becomes:
  \begin{equation}
    \hat{x}[k+1] = \hat{x}[k] + T_s r u[k] - L e[k].
  \end{equation}
  The estimation error is:
  \begin{equation}
    e[k] = x[k] - \hat{x}[k].
  \end{equation}
  Using the state equation and observer update, one obtains 
  \begin{equation}
    e[k+1] = (1-L)e[k].
  \end{equation}
  Thus the estimator pole is:
  \begin{equation}
    z_{est}(L) = 1 - L.
  \end{equation}
  
  The assignment requires an estimator pole 10x slower than the controller pole.
  Convert the controller pole into continuous-time:
  \begin{equation}
    s_{cl} = \frac{1}{T_s} ln(z_{cl}).
  \end{equation}
  An estimator pole ten times slower has:
  \begin{equation}
    s_{est} = \frac{s_{cl}}{10}.
  \end{equation}
  Convert back into discrete-time:
  \begin{equation}
    z_{est} = e^{s_{est} T_s} = z_{cl}^{1/10}.
  \end{equation}
  Finally:
  \begin{equation}
    L = 1 - z_{est}.
  \end{equation}
  This gain is used during the experiments.

  \paragraph{Experiments (correct vs. wrong initial estimate):}
  Two runs are performed:
  \begin{enumerate}
    \item Correct initial estimate: $\hat{x}[0] = x[0]$.
    \item Incorrect initial estimate: $\hat{x}[0] = x[0] + \Delta x$, with $\Delta x$ several $cm$.
  \end{enumerate}
  In both cases, the measured signal $y[k] = -x[k]$ and the estimated output $\hat{y}[k] = -\hat{x}[k]$ are plotted.

  %%% FIGURE 6: Measured vs. estimated distance with slow estimator, 
  %%% correct initial estimate

  %\begin{figure}[h!]
  %  \centering
  %  \includegraphics[width=0.48\textwidth]{state_feedback_assignment}}
  %  \caption{Cart measuring its distance to a wall.}
  %  \label{fig:state_feedback_assignment}
  %\end{figure}


  %%% FIGURE 7: Same but with incorrect initial estimate
  %%% 

  %\begin{figure}[h!]
  %  \centering
  %  \includegraphics[width=0.48\textwidth]{state_feedback_assignment}}
  %  \caption{Cart measuring its distance to a wall.}
  %  \label{fig:state_feedback_assignment}
  %\end{figure}


  \paragraph{Observed behavior:}

  \textbf{1. Correct initial estimate}
  If $\hat{x}[0] ≈ x[0]$, the estimator error

 
  



  \begin{enumerate}
    \item \textbf{Good initial estimate} $\hat{x}[0]\approx x[0]$.
    \item \textbf{Wrong initial estimate} $\hat{x}[0]=\hat{x}_0^\text{bad}$ (set via \texttt{readValue(5)}).
  \end{enumerate}

  Figures~\ref{fig:est_ctrl_good} and~\ref{fig:est_ctrl_bad} plot measured and estimated distances. The closed-loop poles of the combined system are
  \begin{equation}
    \lambda_{1,2} = \{p_\text{cl}(K_\text{nom}),\ p_\text{est}(L_\text{slow})\},
  \end{equation}
  confirming that the estimator dynamics do not affect the control pole (separation principle).

  \begin{figure}[H]\centering
    \fbox{\parbox{0.8\textwidth}{Placeholder for measured vs.\ estimated position, good initial estimate (images/est\_ctrl\_good.pdf).}}
    \caption{Measured and estimated distance with good initial estimate.}
    \label{fig:est_ctrl_good}
  \end{figure}

  \begin{figure}[H]\centering
    \fbox{\parbox{0.8\textwidth}{Placeholder for measured vs.\ estimated position, wrong initial estimate (images/est\_ctrl\_bad.pdf).}}
    \caption{Measured and estimated distance with wrong initial estimate.}
    \label{fig:est_ctrl_bad}
  \end{figure}

  \noindent \textbf{Observations:}
  \begin{itemize}
    \item With correct initialization, the estimator tracks the measurement and the controller meets the design settling time.
    \item With a wrong $\hat{x}[0]$, the estimator converges slowly by design while the control performance remains acceptable, illustrating that the estimator pole does not alter the control pole.
    \item If faster estimator convergence is required, $L$ can be decreased (more negative) at the cost of more noise, as seen in 2(a).
  \end{itemize}

  \newpage
  \bibliographystyle{plain}
  \bibliography{references}

\end{document}
