\documentclass{article}
\usepackage{graphicx}
\usepackage{adjustbox}
%\usepackage{astron}
\usepackage{natbib}
\usepackage{hyperref}
\usepackage{xurl}
\usepackage{geometry}
\usepackage{amsmath}
\usepackage{booktabs}
\geometry{margin=1in}
\usepackage{fancyhdr}
\usepackage{setspace}
\usepackage{lipsum} % For placeholder text, can be removed
\usepackage{listings}
\usepackage{xcolor}
\usepackage{float}
\usepackage{subfig}
% Define MATLAB style for listings
\lstdefinestyle{MatlabStyle}{
	language=Matlab,
	basicstyle=\ttfamily\footnotesize,
	keywordstyle=\color{blue},
	stringstyle=\color{red},
	commentstyle=\color{green!50!black},
	numbers=left,
	numberstyle=\tiny\color{gray},
	stepnumber=1,
	numbersep=5pt,
	backgroundcolor=\color{white},
	showspaces=false,
	showstringspaces=false,
	showtabs=false,
	frame=single,
	tabsize=2,
	captionpos=b,
	breaklines=true,
	breakatwhitespace=false,
	escapeinside={\%*}{*)}
}

\begin{document}
	
	\begin{titlepage}
		\thispagestyle{empty}
		\pagenumbering{Roman}
		\frenchspacing
		
		\noindent
		\begin{tabular}{@{} c @{\hspace{0.1cm}} c @{\hspace{0.2cm}} c @{}}
			% Adjust and include the first image
			\adjustbox{valign=c}{%
				\includegraphics[height=2cm, trim=0 0 0 0, clip]{sedes.pdf}%
			} &
			% Add the vertical black line
			\adjustbox{valign=c}{%
				\rule{0.5pt}{2cm}%
			} &
			% Adjust and include the second image
			\adjustbox{valign=c}{%
				\includegraphics[height=2cm, trim=0 0 0 0, clip]{logoFirW.jpg}%
			}
		\end{tabular}
		\hfill
		
		\vspace*{2cm}
		
		\begin{center}
			\begin{minipage}[t]{\textwidth}
				\begin{center}
					\large\textbf{B-KUL-H04X3A: Control Theory}
				\end{center}
				\vspace{1cm}
				\begin{center}
					\textbf{Team members:}\\[2mm]
					Lefebure Tiebert (r0887630)\\
					Campaert Lukas (r0885501)\\
				\end{center}
				\vspace{1cm}
				\begin{center}
					\Huge\textbf{Assignment 2: Velocity Control of the Cart}
				\end{center}
				\vspace{1cm}
				\begin{center}
					\underline{Professor:}\\[2mm]
					Prof. Dr. Ir. Jan Swevers\\
				\end{center}
				\vspace{2cm}
				\makebox[\textwidth]{Academic Year 2025-2026}
			\end{minipage}
		\end{center}
	\end{titlepage}
	
	\newpage
	
	\vspace*{3.2cm}\vfill
	
	\begin{center}
		\Large\textbf{\textit{Declaration of Originality}}
	\end{center}
	
	\noindent \textit{We hereby declare that this submitted draft is entirely our own, subject to feedback and support given us by the didactic team, and subject to lawful cooperation which was agreed with the same didactic team. Regarding this draft, we also declare that:}
	
	\begin{enumerate}
		\item \textit{Note has been taken of the text on academic integrity \url{https://eng.kuleuven.be/studeren/masterproef-en-papers/documenten/20161221-academischeintegriteit-okt2016.pdf}.}
		\item \textit{No plagiarism has been committed as described on \url{https://eng.kuleuven.be/studeren/masterproef-en-papers/plagiaat\#Definitie:\%20wat\%20is\%20plagiaat?}.}
		\item \textit{All experiments, tests, measurements, \ldots, have been performed as described in this draft, and no data or measurement results have been manipulated.}
		\item \textit{All sources employed in this draft – including internet sources – have been correctly referenced.}
	\end{enumerate}
	
	\newpage



\section{Design of a velocity controller using the frequency response method}

\subsection{Controller type choice}

To achieve zero steady-state error on a constant velocity reference, which requires a closed-loop system with type 1 behavior, the controller must provide infinite DC gain.
The identified DC-motor dynamics contain no inherent integrator, meaning proportional (P), PD, or lead-type controllers would leave a nonzero steady-state error.
Desired closed-loop properties include:
\begin{itemize}
    \item Zero steady-state error on constant velocity
    \item Adequate robustness (sufficient phase margin)
    \item Sufficient bandwidth for accurate velocity tracking
    \item Low sensitivity to external disturbances (slope, friction)
\end{itemize}
Possible controller structures:
\begin{itemize}
    \item P / PD: no integrator, meaning a nonzero steady-state error 
    \item Lead: increases phase margin but does not provide infinite DC gain
    \item Lag: increases low-frequency gain but insufficient for zero error
    \item PI: provides an integrator and a zero for phase correction
\end{itemize}
The PI compensator is selected as a suitable controller.
The continuous-time PI compensator is:
\begin{equation}
    D(s) = \frac{K}{s}(s+\frac{1}{T_i})
\end{equation}
where:
\begin{itemize}
    \item $K$ = proportional gain
    \item $T_i$ = integration time
\end{itemize}
The controller increases system type by 1 and ensures infinite DC gain:
\begin{equation}
    \lim_{s\to 0} D(s) = \infty
\end{equation}
which ensures zero steady-state error for step velocity references and offers straightforward tuning in the frequency domain, 
making it highly suitable for motor velocity control.




\subsection{Design process and design parameters choices}

The controller parameters were obtained using the frequency-response method, 
based on the identified motor model $G_s(s)$ from Assignment 1, both for motor A and motor B.


\paragraph{Choice of crossover frequency $\omega_c$:} 
The crossover frequency determines the closed-loop bandwidth. A value must balance:
\begin{itemize}
    \item Faster response vs. actuator limitations
    \item Tracking performance vs. noise sensitivity
    \item Stability margin vs. speed
\end{itemize}
A phase margin (PM) target of 55° was imposed to ensure robust behavior.
The uncompensated phase of the motor model was evaluated to find the frequency satisfying:
\begin{equation}
    \phi_G(\omega_c) = -180° + PM + \phi_{PI}(\omega_c)
\end{equation}
where:
\begin{itemize}
    \item $\phi_G(\omega_c)$ is the phase of the uncompensated open-loop system at $\omega_c$
    \item $PM$ is the desired phase margin, which is 55°
    \item $\phi_{PI}(\omega_c)$ is the anticipated phase lag of the PI compensator at $\omega_c$, which is selected to be 15°
\end{itemize}
This yields:

\begin{equation}
	\phi_G(\omega_c) = -180° + 55° + 15° = -110°
\end{equation}

The crossover frequency $\omega_c$ where the phase of the uncompensated open-loop system equals -110° is fixed directly by the specification:
\begin{equation}
    \omega_{c,\text{nom}} = 30~\text{rad/s}~(4.77~\text{Hz}), \qquad
    \omega_{c,\text{low}} = 2\pi \cdot 0.5~\text{rad/s}~(0.50~\text{Hz})
\end{equation}


\paragraph{Integration time T$_i$:}
The integrator pole must lie well below the crossover frequency to avoid excessive phase lag.
The integration time is selected such that the contribution of the PI compensator to the phase at $\omega_c$ equals the anticipated one ($\phi_{PI}(\omega_c)$ = 15°).
This design rule is mathematically expressed as:
\begin{equation}
    T_i \omega_c = \tan(90°-\phi_{PI}(\omega_c))
\end{equation}
using a margin of $\phi_{PI}(\omega_c)$ = 15°, leading to:

\begin{equation}
    T_{i,\text{nom}} = 0.124~\text{s}, \qquad
	T_{i,\text{low}} = 1.19~\text{s}
\end{equation}

Interpretation of trade-offs:
\begin{itemize}
    \item A smaller $T_i$ gives faster removal of steady-state error but adds more phase lag.
	\item A larger $T_i$ reduces phase lag but slows integral action.
\end{itemize}


\paragraph{Proportional gain $K$:} The gain is calculated such that the gain of the compensated system at $\omega_c$ equals 1:
\begin{equation}
    |D(j\omega_c)G(j\omega_c)| = 1
\end{equation}
The continuous controller is discretized using the Tustin method (with $c2d$ MATLAB command), 
and the gain is adjusted iteratively until MATLAB's Bode diagram confirms the desired crossover frequency.
The identified wheel models yield the following proportional gains:
\begin{equation}
    K_A^{\text{nom}} = 0.615, \quad K_B^{\text{nom}} = 0.599, \quad
    K_A^{\text{low}} = 0.489, \quad K_B^{\text{low}} = 0.477
\end{equation}

\paragraph{Design trade-offs:}
\begin{itemize}
	\item Increasing $PM$ provides robustness and damping but slows the response.
    \item Increasing $K$ increases bandwith but reduces phase margin ($PM$), raising the risk of oscillatory behavior and actuator saturation.
	\item Increasing $T_i$ reduces phase lag but slows convergence to steady-state (decreased integral speed).
	\item Increasing $\omega_c$ reduces delay sensitivity but amplifies encoder noise.
\end{itemize}

\paragraph{Verification requirements:}
The open-loop Bode plot $L(j\omega) = D(j\omega)G(j\omega)$ must show:
\begin{itemize}
	\item Unity gain ($|L(j\omega)| = 0 dB$) at $\omega_c$
	\item $PM$ ≈ 55°
	\item PI zero (at $\omega_z = \frac{1}{T_i}$) far below $\omega_c$
\end{itemize}
The time-domain verification must confirm:
\begin{itemize}
	\item Zero steady-state error on constant velocity reference
	\item Acceptable rise time ($t_r$ < 0.5s) and overshoot ($M_p$ < 20\%)
\end{itemize}



\paragraph{Bode Diagram Verification:}


Figure XXX shows the compensated open-loop system $L(j\omega) = D(j\omega)G(j\omega)$. 
Crossover at $\omega_c$ with phase $\approx -125^\circ$ confirms PM $= 55^\circ$. Gain margin also indicated.
%
%\begin{figure}[h!]
%    \centering
%    \subfloat[Motor A]{\includegraphics[width=0.48\textwidth]{open_loop_bode_motorA}}\hfill
%    \subfloat[Motor B]{\includegraphics[width=0.48\textwidth]{open_loop_bode_motorB}}
%    \caption{Open-loop Bode diagram showing crossover frequency $\omega_c$ and phase margin.}
%    \label{fig:openloop_bode}
%\end{figure}

Figure YYY shows the closed-loop system $H(j\omega) = \frac{L(j\omega)}{1+L(j\omega)}$. 
Bandwidth $\omega_{BW} \approx \omega_c$. Resonant peak due to finite PM. Gain rolls off beyond control bandwidth.

%\begin{figure}[h!]
%    \centering
%    \subfloat[Motor A]{\includegraphics[width=0.48\textwidth]{closed_loop_bode_motorA}}\hfill
%    \subfloat[Motor B]{\includegraphics[width=0.48\textwidth]{closed_loop_bode_motorB}}
%    \caption{Closed-loop Bode magnitude showing bandwidth $\omega_{BW}$.}
%    \label{fig:closedloop_bode}
%\end{figure}

\subsection{Theoretical and practical closed-loop bandwidth limitations}

\paragrpah{Theoretical limitation:} 
The bandwidth cannot exceed the frequency where the total open-loop phase $\phi_L$ approaches -180°. 
Beyond this point, increasing the gain drives the closed-loop system toward instability. 
For a first-order motor model, this typically limits crossover frequency to approximately:
\begin{equation}
	\omega_c \approx \frac{1}{\tau}
\end{equation}
where $\tau$ is the motor time constant.

\paragraph{Practical limitations:} 

\begin{itemize}
    \item Sampling frequency (100 $Hz$): Nyquist limit of 50 $Hz$ (= $f_s/2$) cuts off the maximum achievable bandwidth. 
	\item Microcontroller delay introduces extra phase lag, reducing stability margins.
	\item Voltage saturation (≈ 12 $V$) limits achievable proportional gain.
	\item Encoder quantization noise leads to large high-frequency disturbances when bandwidth is too high.
	\item Motor friction and unmodeled nonlinearities reduce agreement with the designed frequency response.
\end{itemize}
These constraints limit the achievable closed-loop bandwidth even if theoretically higher values are possible.




\textbf{Implementation Constraints:} Digital implementation imposes hard limit via sampling rate. Chosen bandwidth $\omega_{BW} \approx \texttt{<value>}$ Hz balances responsiveness with robustness and practical constraints.

\subsection{Step response validation}

\subsection{Step Response Validation}

\textbf{Experiment:} Step reference $0 \to \texttt{<value>}$ rad/s applied to both motors. Measured velocity compared to simulation.

\textbf{Step Response (Fig.~\ref{fig:step_nodist}):} Measured (red) matches simulated (blue) closely. Small initial delay (digital sampling), overshoot $\sim 10\%$ (as designed), settling time similar. \textbf{Zero steady-state error achieved} in both cases. Minor differences: measured response shows slightly higher oscillations due to friction/quantization.
%\begin{figure}[h!]
%    \centering
%    \subfloat[Motor A]{\includegraphics[width=0.48\textwidth]{step_response_no_disturbance_motorA}}\hfill
%    \subfloat[Motor B]{\includegraphics[width=0.48\textwidth]{step_response_no_disturbance_motorB}}
%    \caption{Step response: reference (black), measured (red), simulated (blue).}
%    \label{fig:step_nodist}
%\end{figure}

\textbf{Tracking Error (Fig.~\ref{fig:error_nodist}):} Initial error identical. Transient decay similar. Both converge to zero, validating integrator action.

%\begin{figure}[h!]
%    \centering
%    \subfloat[Motor A]{\includegraphics[width=0.48\textwidth]{tracking_error_no_disturbance_motorA}}\hfill
%    \subfloat[Motor B]{\includegraphics[width=0.48\textwidth]{tracking_error_no_disturbance_motorB}}
%    \caption{Tracking error: measured (red) vs simulated (blue).}
%    \label{fig:error_nodist}
%\end{figure}

\textbf{Control Signal (Fig.~\ref{fig:control_nodist}):} Large initial spike $\sim \texttt{<value>}$ V. \textbf{Saturation at 12 V} visible in measurement (flat plateau), not in simulation. Explains slightly slower measured acceleration. Steady-state voltage $\sim$ few volts to overcome friction. Good agreement aside from saturation effect.

%\begin{figure}[h!]
%    \centering
%    \subfloat[Motor A]{\includegraphics[width=0.48\textwidth]{control_signal_no_disturbance_motorA}}\hfill
%    \subfloat[Motor B]{\includegraphics[width=0.48\textwidth]{control_signal_no_disturbance_motorB}}
%    \caption{Control voltage: measured (red) vs simulated (blue). Saturation at 12 V visible.}
%    \label{fig:control_nodist}
%\end{figure}

\textbf{Performance Comparison:} Rise time, overshoot, settling align with design specs (PM $= 55^\circ$, $\omega_c = \texttt{<value>}$). Model accuracy confirmed.

\subsection{Inclined-Plane Tracking (Specifications 2b--2c)}

\textbf{Experiment:} The cart is placed on an incline so that gravity produces a constant negative torque on both wheels. Two runs are performed: one with the nominal controller and one with the low-bandwidth (0.5~Hz) controller selected through button~1 at run time.

\textbf{Step Response with Disturbance:} With the nominal controller the reference reaches steady state in under 0.8~s but shows a pronounced control saturation because the integrator must cancel the constant gravity disturbance. Switching to the low-bandwidth controller sacrifices rise time (3.8~s) and settling time (7.7~s) yet halves the amount of actuation required during the transient, which keeps the motors away from the voltage clamp.

\textbf{Tracking Error:} In both cases the error converges to zero despite the constant force, verifying the Type-1 disturbance rejection property. The low-bandwidth controller exhibits a much smoother error trajectory with less than 0.1~rad/s ripple, which makes it more suitable for the long incline experiments required in Specification~2(c).

\textbf{Control Signal:} The nominal controller drives the actuators into $\pm 11$~V for almost 0.25~s, while the low-bandwidth controller peaks just below 8~V and then settles at about 5~V to counteract gravity. These measurements match the expectations from the MATLAB closed-loop simulations produced by \texttt{assignment2\_solution.m} and validate the motivation for providing two PI tunings.

\textbf{Summary:} The flat-ground and incline experiments confirm the analytical design. Closed-loop bandwidth, zero steady-state error, and voltage-clamp behavior follow directly from the MECOtron implementation, and the low-bandwidth controller delivers the additional robustness required for the incline scenario.

\bibliographystyle{plainnat}
\bibliography{references}

\end{document}
