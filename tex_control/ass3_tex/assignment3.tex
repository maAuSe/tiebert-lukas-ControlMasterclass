\documentclass{article}
\usepackage{graphicx}
\usepackage{adjustbox}
\usepackage{natbib}
\usepackage{hyperref}
\usepackage{xurl}
\usepackage{geometry}
\usepackage{amsmath}
\usepackage{booktabs}
\usepackage{fancyhdr}
\usepackage{setspace}
\usepackage{listings}
\usepackage{xcolor}
\usepackage{float}
\usepackage{subfig}
\geometry{margin=1in}

% Define MATLAB style for listings
\lstdefinestyle{MatlabStyle}{
  language=Matlab,
  basicstyle=\ttfamily\footnotesize,
  keywordstyle=\color{blue},
  stringstyle=\color{red},
  commentstyle=\color{green!50!black},
  numbers=left,
  numberstyle=\tiny\color{gray},
  stepnumber=1,
  numbersep=5pt,
  backgroundcolor=\color{white},
  showspaces=false,
  showstringspaces=false,
  showtabs=false,
  frame=single,
  tabsize=2,
  captionpos=b,
  breaklines=true,
  breakatwhitespace=false,
  escapeinside={\%*}{*)}
}

\begin{document}

  \begin{titlepage}
    \thispagestyle{empty}
    \pagenumbering{Roman}
    \frenchspacing

    \noindent
    \begin{tabular}{@{} c @{\hspace{0.1cm}} c @{\hspace{0.2cm}} c @{}}
      \adjustbox{valign=c}{%
        \includegraphics[height=2cm, trim=0 0 0 0, clip]{sedes.pdf}%
      } &
      \adjustbox{valign=c}{%
        \rule{0.5pt}{2cm}%
      } &
      \adjustbox{valign=c}{%
        \includegraphics[height=2cm, trim=0 0 0 0, clip]{logoFirW.jpg}%
      }
    \end{tabular}
    \hfill

    \vspace*{2cm}

    \begin{center}
      \begin{minipage}[t]{\textwidth}
        \begin{center}
          \large\textbf{B-KUL-H04X3A: Control Theory}
        \end{center}
        \vspace{1cm}
        \begin{center}
          \textbf{Team 47:}\\[2mm]
          Lefebure Tiebert (r0887630)\\
          Campaert Lukas (r0885501)\\
        \end{center}
        \vspace{1cm}
        \begin{center}
          \Huge\textbf{Assignment 3: State feedback and state estimation}
        \end{center}
        \vspace{1cm}
        \begin{center}
          \underline{Professor:}\\[2mm]
          Prof. Dr. Ir. Jan Swevers\\
        \end{center}
        \vspace{2cm}
        \makebox[\textwidth]{Academic Year 2025-2026}
      \end{minipage}
    \end{center}
  \end{titlepage}

  \newpage
  \vspace*{3.2cm}\vfill

  \begin{center}
    \Large\textbf{\textit{Declaration of Originality}}
  \end{center}

  \noindent \textit{We hereby declare that this submitted draft is entirely our own, subject to feedback and support given us by the didactic team, and subject to lawful cooperation which was agreed with the same didactic team. Regarding this draft, we also declare that:}

  \begin{enumerate}
    \item \textit{Note has been taken of the text on academic integrity \url{https://eng.kuleuven.be/studeren/masterproef-en-papers/documenten/20161221-academischeintegriteit-okt2016.pdf}.}
    \item \textit{No plagiarism has been committed as described on \url{https://eng.kuleuven.be/studeren/masterproef-en-papers/plagiaat\#Definitie:\%20wat\%20is\%20plagiaat?}.}
    \item \textit{All experiments, tests, measurements, \ldots, have been performed as described in this draft, and no data or measurement results have been manipulated.}
    \item \textit{All sources employed in this draft --- including internet sources --- have been correctly referenced.}
  \end{enumerate}

  \newpage

\setcounter{page}{1}
\pagenumbering{arabic}

\section{State estimator and state feedback controller design using pole placement}

The goal of this assignment is to control the position of the cart along a straight line. The state is the cart position $x(t)$ [m] along the $x$-axis, with the wall at the origin $x$ = 0 m. 
The cart is positioned in front of the wall, so $x<0$. The infrared sensor measures the positive distance to the wall, \ i.e. it measures $-x$.\newline
  
\noindent The system parameters are:
\begin{itemize}
    \item Wheel radius: $r = 0.033$~m
    \item Sampling time: $T_s = 0.01$~s (100~Hz control loop)
\end{itemize}

\noindent The input $u(t) = \omega(t)$ [rad/s] is the common wheel angular velocity setpoint, applied equally to both motors A and B via the inner velocity controllers designed in Assignment~2. 
The velocity control loop is assumed ideal, meaning the actual wheel velocity tracks the setpoint perfectly.

\subsection{Discrete-time state equation (1a)} % Section 1(a)
  
\paragraph{Continuous-time state equation:} 
The velocity control loop from Assignment~2 is assumed ideal, so the actual cart velocity equals the commanded velocity. The continuous-time kinematic relation is:
\begin{equation}
    \dot{x}(t) = v(t) = r \cdot \omega(t) = r \cdot u(t),
\end{equation}
where $v(t)$ [m/s] is the linear cart velocity and $u(t) = \omega(t)$ [rad/s] is the wheel angular velocity input.\\
  
\noindent In state-space form \cite{CT-slides-C9} with scalar state $x(t)$ [m] and input $u(t)$ [rad/s]:
\begin{equation}
    \dot{x}(t) = A_c x(t) + B_c u(t), \quad A_c = [0],\quad B_c = [r] = [0.033\; \text{m}].
\end{equation}

\paragraph{Forward Euler discretization:} 
The forward Euler method approximates the derivative as:
\begin{equation}
    \dot{x}[k] \approx \frac{x[k+1] - x[k]}{T_s}.
\end{equation}
Substituting into the continuous-time equation:
\begin{equation}
    \frac{x[k+1] - x[k]}{T_s} = A_c x[k] + B_c u[k] = 0 \cdot x[k] + r \cdot u[k].
\end{equation}
Solving for $x[k+1]$:
\begin{equation}
    x[k+1] = x[k] + T_s \cdot r \cdot u[k].
\end{equation}
  
\noindent The discrete-time state-space model is:
\begin{equation}
    x[k+1] = A_d x[k] + B_d u[k],
\end{equation}
with the discrete-time system matrices:
\begin{equation}
    \boxed{A_d = [1], \qquad B_d = [T_s \cdot r] = 0.01 \times 0.033\ = 3.3 \times 10^{-4}\ \text{m/(rad/s)}.}
\end{equation}
  
\noindent Note: In the state estimator implementation, the measured average wheel speed $u[k] = (\omega_A[k] + \omega_B[k])/2$ is used for the prediction step.


\subsection{Measurement equation (1b)} % Section 1(b)

The front infrared (IR) sensor measures the distance from the cart to the wall. Since the cart is at position $x < 0$ (in front of the wall at $x$ = 0 m), the measured distance is a positive quantity equal to $|x| = -x$.
  
\noindent The measurement equation (ignoring noise) is:
\begin{equation}
    y[k] = -x[k].
\end{equation}
  
\noindent In state-space form:
\begin{equation}
    y[k] = C x[k] + D u[k],
\end{equation}
\noindent with the output matrices:
\begin{equation}
    \boxed{C = [-1], \qquad D = [0].}
\end{equation}
  
\noindent Here $C = [-1]$ because the sensor measures the negation of the state, and $D = [0]$ since the input does not directly affect the measurement.


\subsection{Design of state feedback controller gain $K$ using pole placement (1c)} % Section 1(c)
  
The position controller outputs the desired wheel angular velocity based on the position error. Assuming full state feedback (no estimator), the control law is:
\begin{equation}
    u[k] = K \cdot (x_\text{ref}[k] - x[k]),
\end{equation}
where $x_\text{ref}[k]$ [m] is the reference position, $x[k]$ [m] the measured position, $K$ [rad/(s$\cdot$m)] the state feedback gain, and $u[k]$ [rad/s] the commanded wheel angular velocity.

\paragraph{Derivation of closed-loop pole as function of $K$:}
Substituting the control law into the discrete-time state equation:
\begin{align}
    x[k+1] &= A_d x[k] + B_d u[k] \\
           &= x[k] + T_s r \cdot K (x_\text{ref}[k] - x[k]) \\
           &= (1 - T_s r K) x[k] + T_s r K \cdot x_\text{ref}[k].
\end{align}
  
\noindent The closed-loop system has the form:
\begin{equation}
    x[k+1] = A_\text{cl} x[k] + B_\text{cl} x_\text{ref}[k],
\end{equation}
where the closed-loop system matrix is:
\begin{equation}
    \boxed{A_\text{cl} = [1 - T_s r K].}
\end{equation}
  
\noindent For this first-order system, the closed-loop pole equals the system matrix:
\begin{equation}
    \boxed{z_\text{cl}(K) = 1 - T_s r K = 1 - 0.01 \times 0.033 \times K.}
\end{equation}

\paragraph{Pole behavior as function of $K$:}
At $K = 0$, $z_\text{cl} = 1$ (marginally stable integrator). As $K$ increases, the pole moves left along the real axis, reaching $z_\text{cl} = 0$ ("deadbeat") at $K \approx 3030$ rad/(s$\cdot$m) and the stability boundary $z_\text{cl} = -1$ at $K \approx 6060$ rad/(s$\cdot$m).

\paragraph{Stability analysis:}
For discrete-time stability, the pole must lie inside the unit circle:
\begin{equation}
    |z_\text{cl}(K)| = |1 - T_s r K| < 1.
\end{equation}
This yields the stability condition:
\begin{equation}
    \boxed{0 < K < \frac{2}{T_s r} = \frac{2}{0.01 \cdot 0.033} \approx 6060\ \text{rad/(s$\cdot$m)}.}
\end{equation}
  
\noindent This means the system can become unstable if $K$ $>$ 6060 \text{rad/(s$\cdot$m)}, causing the pole to exit the unit circle through $z = -1$.

\paragraph{Pole-zero map:}
Figure~\ref{fig:pole_map_K} shows the closed-loop pole location for varying $K$. The pole moves along the real axis from $z = 1$ (at $K = 0$) toward $z = -1$ (at $K = K_\text{max} \approx$ 6060 rad/(s$\cdot$m)).

\begin{figure}[H]\centering
    \includegraphics[width=0.7\textwidth]{images/pole_map_K.pdf}
    \caption{Closed-loop pole location $z_\text{cl}(K) = 1 - T_s r K$ for varying $K$. Unit circle is shown for stability reference.}
    \label{fig:pole_map_K}
\end{figure}

\paragraph{Simulated step responses:}
The discrete-time step response is simulated for $K \in \{20, 40, 80\}$ rad/(s$\cdot$m):
\begin{equation}
    x[k+1] = (1 - T_s r K) x[k] + T_s r K \cdot x_\text{ref}, \quad x[0] = x_0.
\end{equation}
  
\begin{table}[H]\centering
    \begin{tabular}{lccc}
      \toprule
      $K$ [rad/(s$\cdot$m)] & $z_\text{cl}$ & Pole location & Expected behavior \\
      \midrule
      20  & 0.9934 & close to 1 & slow convergence, long settling time \\
      40  & 0.9868 & moderate   & faster response, good compromise \\
      80  & 0.9736 & closer to 0 & fast response, higher control effort \\
      \bottomrule
    \end{tabular}
    \caption{Closed-loop poles for different $K$ values.}
    \label{tab:K_poles}
\end{table}

\begin{figure}[H]\centering
    \includegraphics[width=0.8\textwidth]{images/step_K_sweep.pdf}
    \caption{Simulated closed-loop step responses. Larger $K$ yields faster convergence but higher peak velocity commands.}
    \label{fig:step_K_sweep}
\end{figure}

\paragraph{Relationship between pole location, $K$, and time response:}
\begin{itemize}
    \item for \textbf{small $K$}, e.g. $K_{slow} = 20$ rad/(s$\cdot$m) ($z_\text{cl} \approx 1$): slow exponential decay, long settling time, smooth but sluggish response
    \item for \textbf{moderate $K$}, e.g. $K_{mod} = 40$ rad/(s$\cdot$m) ($0 < z_\text{cl} < 1$): faster convergence, monotonic response without oscillation
    \item for \textbf{large $K$}, e.g. $K_{fast} = 80$ rad/(s$\cdot$m) ($z_\text{cl} < 0$): sign-alternating (oscillatory) discrete-time response; if $|z_\text{cl}| < 1$, still stable but with overshoot
    \item for $K > K_\text{max} \approx 6060$ rad/(s$\cdot$m) ($|z_\text{cl}| > 1$): unstable, diverging oscillations
\end{itemize}

\paragraph{Choice of $K$:}
The theoretical stability limit is $K_\text{max} \approx 6060$ rad/(s$\cdot$m), but practical constraints are more restrictive:
\begin{itemize}
    \item actuator saturation: the motor voltage is limited to $\approx$ 11--12~V; large $K$ causes large velocity commands for small position errors, potentially saturating the inner velocity loop
    \item sensor noise: high $K$ amplifies measurement noise into the control signal
    \item IR sensor range: the sensor is accurate only for 5--30~cm; aggressive control may drive the cart outside this range
\end{itemize}
  
\paragraph{Selected value:}
\begin{equation}
    \boxed{K_{nom} = 40\ \text{rad/(s$\cdot$m)}.}
\end{equation}
This gives $z_\text{cl} = 1 - 0.01 \times 0.033 \times 40 = 0.9868$, providing:
\begin{itemize}
    \item stable operation well within the stability margin,
    \item reasonable settling time $t_s$ ($\approx$ 3 s to reach and stay within $2\%$ of the final steady-state value),
    \item control effort within actuator limits for typical position steps (0.1--0.2~m).
\end{itemize}




\subsection{Design of state estimator gain $L$ using pole placement (1d)} % Section 1(d)
  
A discrete-time Luenberger observer is used to estimate the cart position from the IR sensor measurement:
\begin{equation}
    \hat{x}[k+1] = A_d \hat{x}[k] + B_d u[k] + L \cdot \nu[k],
\end{equation}
where $\nu[k] = y[k] - C\hat{x}[k]$ is the innovation (measurement prediction error). \\
With $A_d = [1]$, $B_d = [T_s r]$, and $C = [-1]$:
\begin{equation}
    \hat{x}[k+1] = \hat{x}[k] + T_s r \cdot u[k] + L \cdot (y[k] + \hat{x}[k]).
\end{equation}

\paragraph{Derivation of estimator pole as function of $L$:}
Define the estimation error:
\begin{equation}
    e[k] = x[k] - \hat{x}[k].
\end{equation}
The true state evolves as $x[k+1] = x[k] + T_s r \cdot u[k]$. Subtracting the observer equation:
\begin{align}
    e[k+1] &= x[k+1] - \hat{x}[k+1] \\
           &= (x[k] + T_s r \cdot u[k]) - (\hat{x}[k] + T_s r \cdot u[k] + L(y[k] - C\hat{x}[k])) \\
           &= (x[k] - \hat{x}[k]) - L(Cx[k] - C\hat{x}[k]) \\
           &= e[k] - L \cdot C \cdot e[k] \\
           &= (1 - LC) \cdot e[k].
\end{align}
Substituting $C = [-1]$:
\begin{equation}
    e[k+1] = (1 - L \cdot (-1)) \cdot e[k] = (1 + L) \cdot e[k].
\end{equation}
The estimator error dynamics have the closed-loop pole:
\begin{equation}
    \boxed{z_\text{est}(L) = 1 + L.}
\end{equation}

\paragraph{Pole behavior as function of $L$:}
\begin{itemize}
    \item at $L = 0$: $z_\text{est} = 1$ (no correction, estimator ignores measurements)
    \item as $L$ becomes more negative: pole moves left along the real axis
    \item at $L = -1$: $z_\text{est} = 0$ (deadbeat estimator, instant convergence)
    \item at $L = -2$: $z_\text{est} = -1$ (stability boundary)
\end{itemize}

\paragraph{Stability analysis:}
For the estimator to be stable:
\begin{equation}
    |z_\text{est}(L)| = |1 + L| < 1.
\end{equation}
This yields:
\begin{equation}
    \boxed{-2 < L < 0.}
\end{equation}
This means the estimator can become unstable:
\begin{itemize}
    \item if $L > 0$: pole $z_\text{est} > 1$, estimation error grows exponentially
    \item if $L < -2$: pole $z_\text{est} < -1$, estimation error diverges with oscillations
\end{itemize}

\paragraph{Trade-offs in pole placement:}
\begin{itemize}
    \item faster estimator (larger $|L|$, pole closer to 0):
\begin{itemize}
      \item [+] faster convergence from wrong initial estimate
      \item [+] quicker correction of model errors
      \item [$-$] higher sensitivity to measurement noise (noise is amplified by $L$)
      \item [$-$] risk of oscillatory behavior if $z_\text{est} < 0$
\end{itemize}
    \item slower estimator (smaller $|L|$, pole closer to 1):
\begin{itemize}
      \item[+] smoother estimate, better noise rejection
      \item[$-$] slow convergence from initialization errors
      \item[$-$] poor tracking of rapid state changes
    \end{itemize}
\end{itemize}
  
\paragraph{Design guideline:} The estimator pole should typically be 2--6 times faster than the :, ensuring the estimation error decays before significantly affecting control performance.

\paragraph{Pole placement and trade-offs:} The state estimator gain $L$ is selected such that the estimator is $3$--$5$ times faster than the controller for tasks in Section~2.1--2.2, then set it 10 times slower than the controller for Section~2.3 per the specification. \\
The practical limits are set by sensor noise and discretization: $|1+L|<1\Rightarrow -2<L<0$.

\paragraph{Chosen gain:} A nominal value:
\begin{equation}
    \boxed{L_{nom}=-0.18}
\end{equation}
gives the closed-loop pole
\begin{equation}
  z_\text{est}= 1 + L_{nom} = 0.82  
\end{equation}
which is fast but noise-aware. Alternative gains $L\in\{-0.05,-0.18,-0.35\}$ are swept in the experiments below.



  
\section{Implementation and testing of state estimator and state feedback controller}
  
\subsection{Estimator only: wrong initial estimate, different $L$ (2a)} % Section 2(a)
The controller is disabled; the estimator starts from a wrong initial position $\hat{x}[0]$.
\\ \\
Configuration:
\begin{itemize}
    \item estimator active, controller off; the cart is moved manually, or driven with a known, simple input signal
    \item the state estimator is initialized with a wrong position estimate $\hat{x}[0]$ (e.g. 10 cm closer to the wall than reality).
    \item different values of $L$ are tested (e.g. $L \in$ \{$-0.05, -0.18, -0.35$\}).
\end{itemize}
For each L,  the measured distance $y[k]$ (positive, measured by IR sensor) and the estimated state $\hat{x}[k]$ are logged. Figure~\ref{fig:estimator_L_sweep} overlays the measured distance $y=-x$ and the estimated distance $-\hat{x}$ for $L\in\{-0.05,-0.18,-0.35\}$. Larger $|L|$ yields faster convergence but more noise on $\hat{x}$ and $\nu$.

\begin{figure}[H]\centering
    \includegraphics[width=0.8\textwidth]{images/estimator_L_sweep.pdf}
    \caption{Measured vs. estimated distance for different $L$ (wrong initial estimate).}
    \label{fig:estimator_L_sweep}
\end{figure}

\paragraph{Interpretation and link to design of state estimator gain $L$:}
As shown analytically, the closed-loop pole is
\begin{equation}
    z_\text{est}(L) = 1 + L.
\end{equation}
For \textbf{small $|L|$} (e.g. $L \approx -0.05$, so $z_\text{est} \approx 0.95$):
\begin{itemize}
    \item the estimator reacts slowly to the discrepancy between measurement and prediction
    \item the estimate converges slowly to the measurement, a substantial transient remains for a long time
    \item this matches the formula $z_{est} = 1 + L$: pole close to 1 leads to slow dynamics
\end{itemize}
For \textbf{moderate $|L|$} (e.g. the designed $L_{nom} \approx -0.18$, giving $z_{est} \approx 0.82$):
\begin{itemize}
    \item the estimate converges significantly faster to the measurement
    \item noise on the measurement is visible but not overly amplified
\end{itemize}
For \textbf{large $|L|$} closer to $-2$ (e.g. $L = -0.35$, so $z_\text{est}$ = $0.65$):
\begin{itemize}
    \item convergence is very fast, but the estimate closely follows measurement noise, becoming noisy and irregular
    \item this shows the trade-off: faster convergence vs. noise sensitivity
\end{itemize}
Summarized:
  \begin{itemize}
    \item convergence speed increases with $|L|$, consistent with $z_\text{est}=1+L$
    \item noise on $\hat{x}$ and the innovation $\nu$ grows with $|L|$, illustrating the trade-off from Section 1.4 (1d)
    \item the chosen $L_\text{nom}$ balances settling time $t_s$ and noise
  \end{itemize}
From the plots, one concludes that larger negative $L$ values (closer to $-2$) give faster convergence of the estimates to the measurements, 
consistent with the expression $z_\text{est} = 1 + L$ and the trade-offs discussed in Section 1.4 (design of state estimator gain $L$).



\subsection{Controller only: proportional feedback for different $K$ (2b)} % Section 2(b)
The estimator is disabled, feedback uses the raw distance $y[k] = -x[k]$. For different values of $K\in\{K_{slow}, K_{nom}, K_{fast}\}$ = \{20, 40, 80\} rad/(s $\cdot m)$, a 0.25 m step reference is applied.
\\ \\
Configuration:
\begin{itemize}
    \item controller active, estimator off; the controller directly uses the IR measurement (converted to position) for feedback
    \item the control law becomes: $u[k] = K(r[k]-x[k])$, with $x[k]$ obtained from the IR sensor relation $x[k] = -y[k]$ 
    \item a 0.25 m step in position reference is applied
\end{itemize}
This experiment is repeated for several values of $K$ (for $K\in\{20,40,80\}$ rad/(s $\cdot m)$).


\noindent Figure~\ref{fig:controller_K_response} shows the step reference (dashed line) and position responses for different values of $K$. Figure~\ref{fig:controller_K_voltage} shows the corresponding motor voltages, for the same three values of $K$.

\begin{figure}[H]\centering
    \includegraphics[width=0.8\textwidth]{images/controller_K_response.pdf}
    \caption{Measured position step responses for different $K$ (estimator disabled).}
    \label{fig:controller_K_response}
\end{figure}

\begin{figure}[H]\centering
    \includegraphics[width=0.8\textwidth]{images/controller_K_voltage.pdf}
    \caption{Motor voltage commands for the responses in Figure~\ref{fig:controller_K_response}.}
    \label{fig:controller_K_voltage}
\end{figure}

\paragraph{Interpretation and link to design of state feedback controller gain $K$:}
As shown analytically, the closed-loop pole is
\begin{equation}
    z_\text{cl}(K) = 1 - T_s r K.
\end{equation}
For \textbf{small $K$} (e.g. $K_{slow}$ = 20 rad/(s$\cdot$m)):
\begin{itemize}
    \item pole $z_{cl}$ close to 1: slow rise, long settling time, very smooth control signal
    \item this matches time-responses with slow approach to the reference
\end{itemize}
For \textbf{moderate $K$} (e.g. the designed $K_{nom}$ = 40 rad/(s$\cdot$m)):
  \begin{itemize}
    \item pole closer to zero: faster response, reduced settling time
    \item acceptable overshoot and control signals well within approximately 11--12 V
    \item good tracking performance
\end{itemize}
For \textbf{large $K$} closer to the theoretical maximum $\frac{2}{T_s r}$ (e.g. $K_{fast}$ = 80 rad/(s$\cdot$m)):
\begin{itemize}
    \item pole approaches -1: discrete-time oscillatory response and possibly overshoot
    \item the control signals reach or exceed saturation ($\approx$ 11--12 V), so the inner velocity loop cannot follow the demanded angular velocity
\end{itemize}
Summarized: 
\begin{itemize}
    \item rise time $t_r$ and steady-state error match the simulated trend from Section 1.3 (design of state feedback controller gain $K$), higher $K$ speeds up the response but increases peak voltage
    \item for $K=K_\text{fast}$ = 80 rad/(s$\cdot$m), the voltage briefly saturates, explaining the mild overshoot; this bounds feasible $K$ in practice
    \item the selected $K_\text{nom}$ = 40 rad/(s$\cdot$m) avoids saturation while providing acceptable settling time $t_s$ and negligible steady-state error
\end{itemize}
These observations correspond to the theoretical dependence $z_{cl}(K) = 1 - T_s r K$: as $K$ increases, the closed-loop pole $z_{cl}$ moves left towards -1, giving faster dynamics until saturation and discrete oscillations become limiting.

\paragraph{Choice of $K$:}
There are both theoretical and practical limits for $K$.
\\ \\
Theoretical limit: stability in discrete-time requires
\begin{equation}
    0 < K < \frac{2}{T_s r} \approx 6060\ \text{rad/(s$\cdot$m)}.
\end{equation}
\\ \\
Practical limits:
\begin{itemize}
    \item the low-level motor voltage is bounded: too large $K$ causes saturation, making the actual behavior deviate from the linear model,
    \item large $K$ amplifies sensor noise and model errors,
    \item the IR sensor has a limited range (approximately 5--30 cm); overshoot may drive the cart outside this range, resulting in invalid measurements.
\end{itemize}
In light of these constraints, a value around
\begin{equation}
    \boxed{K_{nom} = 40\; \text{rad/(s$\cdot$m)}}
\end{equation}
offers a good compromise between tracking performance and actuator/sensor limitations, and is therefore used in the remainder of the assignment.

  



\subsection{Combined estimator and controller: influence of estimator pole choice (2c)} % Section 2(c)
In this experiment, both the state feedback controller and the state estimator are active. The controller uses the estimated position $\hat{x}[k]$ instead of the true position:
\begin{equation}
    u[k] = K(x_\text{ref}[k] - \hat{x}[k]).
\end{equation}
Per the assignment specification, the estimator is designed to be 10 times slower than the controller.
\paragraph{Recap of closed-loop poles:}
 From Sections 1.3 (1c) and 1.4 (1d), the closed-loop poles are:
\begin{itemize}
    \item controller pole: $z_\text{cl}(K) = 1 - T_s r K$,
    \item estimator pole: $z_\text{est}(L) = 1 + L$.
\end{itemize}
With $K = 40$ rad/(s$\cdot$m):
\begin{equation}
    z_\text{cl} = 1 - 0.01 \times 0.033 \times 40 = 0.9868.
\end{equation}
\paragraph{Designing the estimator to be 10$\times$ slower than the controller:}
A discrete-time pole closer to 1 corresponds to slower dynamics. To make the estimator 10 times slower than the controller in continuous-time:
\begin{enumerate}
    \item convert discrete-time controller pole $z_{cl}$ to continuous-time: $s_\text{cl} = \frac{\ln(z_\text{cl})}{T_s} = \frac{\ln(0.9868)}{0.01} \approx -1.33$ rad/s
    \item estimator pole is 10$\times$ slower: $s_\text{est} = \frac{s_\text{cl}}{10} = -0.133$ rad/s
    \item convert back to discrete-time: $z_\text{est} = e^{s_\text{est} \cdot T_s} = e^{-0.00133} \approx 0.9987$
\end{enumerate}
Solving for $L$ using $z_\text{est} = 1 + L$:
\begin{equation}
    L = z_\text{est} - 1 = 0.9987 - 1 = -0.0013.
\end{equation}

\paragraph{Selected slow estimator gain:}
  \begin{equation}
    \boxed{L_\text{slow} = -0.0013\; \text{[$-$]}.}
  \end{equation}
This value for $L$ will be used in the remainder of Section 2.3 (2c) during the experiments.

\paragraph{Full closed-loop system poles:}
The combined system (controller + estimator) has two poles:
\begin{equation}
    \boxed{\lambda_1 = z_\text{cl} = 0.9868, \qquad \lambda_2 = z_\text{est} = 0.9987,}
\end{equation}
which corresponds to
\begin{equation}
    \boxed{K = 40\; \text{rad/(s$\cdot$m)}, \qquad L = -0.0013}
\end{equation}

\paragraph{Experiments (correct vs. wrong initial estimate):}
Two runs are performed:
\begin{enumerate}
    \item correct initial estimate: $\hat{x}[0] = x[0]$.
    \item incorrect initial estimate: $\hat{x}[0] = x[0] + \Delta x$, with $\Delta x$ several cm.
\end{enumerate}
In both cases, the measured signal $y[k] = -x[k]$ and the estimated output $\hat{y}[k] = -\hat{x}[k]$ are plotted.

\begin{figure}[H]\centering
    \includegraphics[width=0.8\textwidth]{images/est_ctrl_good.pdf}
    \caption{Combined estimator and controller with good initial estimate. The estimator tracks the measurement, and the controller achieves the design settling time $t_s$.}
    \label{fig:est_ctrl_good}
\end{figure}

\begin{figure}[H]\centering
    \includegraphics[width=0.8\textwidth]{images/est_ctrl_bad.pdf}
    \caption{Combined estimator and controller with wrong initial estimate. The slow estimator takes a long time to converge, but the control performance remains acceptable.}
    \label{fig:est_ctrl_bad}
\end{figure}




\paragraph{Observed behavior:}
\begin{enumerate}
    \item \textbf{Correct initial estimate:}
        If $\hat{x}[0] \approx x[0]$, the estimator error is nearly zero, so:
        \begin{itemize}
            \item the estimator output $\hat{x}[k]$ is immediately accurate,
            \item the controller behaves almost exactly as in the no-estimator case,
            \item the step response closely matches the ideal closed-loop behavior governed by $z_{cl}$.
        \end{itemize}
    \item \textbf{Incorrect initial estimate:}
            Because the chosen estimator pole is very slow:
            \begin{equation}
                e[k+1] = z_{est} \cdot e[k], \quad \text{with} \quad z_{est} \approx z_{cl}^{1/10} \approx 0.9987,
            \end{equation}
            the error decays only gradually. During this long transient, the controller acts on:
            \begin{equation}
                x_{ref}[k] - \hat{x}[k],
            \end{equation}
            which may be very different from the true distance error. \\
            Consequences of the incorrect initial estimate are:
            \begin{itemize}
                \item delayed motion toward the setpoint,
                \item possible motion in the wrong direction initially,
                \item slow convergence to the correct position,
                \item degraded control performance even though the controller gain $K$ is unchanged.
            \end{itemize}
            This demonstrates that a too-slow estimator harms the transient performance, especially when initial estimation erros are present.
    \item \textbf{When designing a state estimator yourself (using pole placement):}
            Instead of making the estimator slower, the closed-loop estimator pole would be placed faster than the controller pole, 
            typically 2--5 times faster in continuous-time terms. 
            For example, if the controller pole is $z_{cl} = 0.2$, a good choice for the estimator pole is around $z_{est} \approx 0.1$ (or even smaller), which corresponds to an $L$ that gives very quick convergence of $\hat{x}[k]$ without excessive noise amplification.
\end{enumerate}

\paragraph{Full closed-loop system poles:}
The combined controller-estimator system in $[\,x[k],\, \hat{x}[k]\,]^T$ has the update matrix:
\begin{equation}
    A_{\text{full}} = 
    \begin{bmatrix}
    1 - T_s r K & T_s r K \\ 
    -L & 1 - L - T_s r K
    \end{bmatrix}
\end{equation}
The eigenvalues of $A_{full}$ are:
\begin{equation}
    \lambda_{1} = 1 - T_s K = z_{cl} \quad \text{(controller pole)}, \qquad \lambda_{2} = 1 - L C = z_{est} \quad \text{(estimator pole)}.
\end{equation}
The full closed-loop system has two poles: one equal to the controller pole and one equal to the estimator pole.
They appear separately and are not coupled.
This matches the separation principle: for linear systems with full-order observers, the controller and estimator can be designed independently,
and their poles show up unchanged in the combined system:
\begin{itemize}
    \item modifying $K$ moves only $z_{cl}$,
    \item modifying $L$ moves only $z_{est}$.
\end{itemize}
However, the estimator performance does influence the control performance in practice.
Even though the controller pole $z_{cl}$ is unchanged, the controller acts on $\hat{x}[k]$ rather than on the true state $x[k]$. 
If the estimator is very slow or starts from a wrong initial condition, $\hat{x}[k]$ can be far from $x[k]$ for a long time,
and the control input is based on this wrong information. 
The result is a poor transient response (slow, possibly with motion in the wrong direction) even though the poles of the closed-loop system are identical.
\\ \\
The closed-loop transfer function from reference $x_{ref}$ to output $y$ only describes the asymptotic dynamics
assuming the internal states are already consistent (i.e. $\hat{x} \approx x$). 
The poles show how fast modes decay, but they do not encode the effect of wrong initial conditions in the internal estimator.
The estimator performance depends on \textbf{the estimator pole $z_{cl}$ and the initial estimation error} $e[0] = x[0] - \hat{x}[0]$,
which only appears in the time-domain solution, not in the pole values themselves.
Therefore, the poor transient behaviour seen with a slow estimator and wrong $\hat{x}[0]$
cannot be predicted from the closed-loop poles alone. Time-domain simulations or experimental plots of $x[k]$ and $\hat{x}[k]$
are needed to fully assess the control performance.


\paragraph{Conclusion:}
Although both poles $z_{cl}$ and $z_{est}$ lie inside the unit circle, meaning the closed-loop system is formally stable,
choosing $z_{est}$ much slower (e.g. 10 times slower) than $z_{cl}$:
\begin{itemize}
    \item severely reduces closed-loop performance,
    \item introduces long transients when $\hat{x}[0]$ is wrong,
    \item demonstrates why \textbf{estimators are typically placed faster than controllers}, not slower.
\end{itemize}



\newpage
\bibliographystyle{plain}
\bibliography{references-assign-3}

\end{document}