\documentclass{article}
\usepackage{graphicx}
\usepackage{adjustbox}
%\usepackage{astron}
\usepackage{natbib}
\usepackage{hyperref}
\usepackage{xurl}
\usepackage{geometry}
\usepackage{amsmath}
\usepackage{booktabs}
\geometry{margin=1in}
\usepackage{fancyhdr}
\usepackage{setspace}
\usepackage{lipsum} % For placeholder text, can be removed
\usepackage{listings}
\usepackage{xcolor}
\usepackage{float}
\usepackage{subfig}
% Define MATLAB style for listings
\lstdefinestyle{MatlabStyle}{
	language=Matlab,
	basicstyle=\ttfamily\footnotesize,
	keywordstyle=\color{blue},
	stringstyle=\color{red},
	commentstyle=\color{green!50!black},
	numbers=left,
	numberstyle=\tiny\color{gray},
	stepnumber=1,
	numbersep=5pt,
	backgroundcolor=\color{white},
	showspaces=false,
	showstringspaces=false,
	showtabs=false,
	frame=single,
	tabsize=2,
	captionpos=b,
	breaklines=true,
	breakatwhitespace=false,
	escapeinside={\%*}{*)}
}

\begin{document}
	
	\begin{titlepage}
		\thispagestyle{empty}
		\pagenumbering{Roman}
		\frenchspacing
		
		\noindent
		\begin{tabular}{@{} c @{\hspace{0.1cm}} c @{\hspace{0.2cm}} c @{}}
			% Adjust and include the first image
			\adjustbox{valign=c}{%
				\includegraphics[height=2cm, trim=0 0 0 0, clip]{sedes.pdf}%
			} &
			% Add the vertical black line
			\adjustbox{valign=c}{%
				\rule{0.5pt}{2cm}%
			} &
			% Adjust and include the second image
			\adjustbox{valign=c}{%
				\includegraphics[height=2cm, trim=0 0 0 0, clip]{logoFirW.jpg}%
			}
		\end{tabular}
		\hfill
		
		\vspace*{2cm}
		
		\begin{center}
			\begin{minipage}[t]{\textwidth}
				\begin{center}
					\large\textbf{B-KUL-H04X3A: Control Theory}
				\end{center}
				\vspace{1cm}
				\begin{center}
					\textbf{Team members:}\\[2mm]
					Lefebure Tiebert (r0887630)\\
					Campaert Lukas (r0885501)\\
				\end{center}
				\vspace{1cm}
				\begin{center}
					\Huge\textbf{Assignment 2: Velocity Control of the Cart}
				\end{center}
				\vspace{1cm}
				\begin{center}
					\underline{Professor:}\\[2mm]
					Prof. Dr. Ir. Jan Swevers\\
				\end{center}
				\vspace{2cm}
				\makebox[\textwidth]{Academic Year 2025-2026}
			\end{minipage}
		\end{center}
	\end{titlepage}
	
	\newpage
	
	\vspace*{3.2cm}\vfill
	
	\begin{center}
		\Large\textbf{\textit{Declaration of Originality}}
	\end{center}
	
	\noindent \textit{We hereby declare that this submitted draft is entirely our own, subject to feedback and support given us by the didactic team, and subject to lawful cooperation which was agreed with the same didactic team. Regarding this draft, we also declare that:}
	
	\begin{enumerate}
		\item \textit{Note has been taken of the text on academic integrity \url{https://eng.kuleuven.be/studeren/masterproef-en-papers/documenten/20161221-academischeintegriteit-okt2016.pdf}.}
		\item \textit{No plagiarism has been committed as described on \url{https://eng.kuleuven.be/studeren/masterproef-en-papers/plagiaat\#Definitie:\%20wat\%20is\%20plagiaat?}.}
		\item \textit{All experiments, tests, measurements, \ldots, have been performed as described in this draft, and no data or measurement results have been manipulated.}
		\item \textit{All sources employed in this draft – including internet sources – have been correctly referenced.}
	\end{enumerate}
	
	\newpage



\section{Design of a velocity controller using the frequency response method}

\subsection{Controller type choice}

To achieve zero steady-state error on a constant velocity reference, which requires a closed-loop system with type 1 behavior, the controller must provide infinite DC gain.
The identified DC-motor dynamics contain no inherent integrator, meaning proportional (P), PD, or lead-type controllers would leave a nonzero steady-state error.
Desired closed-loop properties include:
\begin{itemize}
    \item Zero steady-state error on constant velocity
    \item Adequate robustness (sufficient phase margin)
    \item Sufficient bandwidth for accurate velocity tracking
    \item Low sensitivity to external disturbances (slope, friction)
\end{itemize}
Possible controller structures:
\begin{itemize}
    \item P / PD: no integrator, meaning a nonzero steady-state error 
    \item Lead: increases phase margin but does not provide infinite DC gain
    \item Lag: increases low-frequency gain but insufficient for zero error
    \item PI: provides an integrator and a zero for phase correction
\end{itemize}
The PI compensator is selected as a suitable controller.
The continuous-time PI compensator is:
\begin{equation}
    D(s) = \frac{K}{s}(s+\frac{1}{T_i})
\end{equation}
where:
\begin{itemize}
    \item $K$ = proportional gain
    \item $T_i$ = integration time
\end{itemize}
The controller increases system type by 1 and ensures infinite DC gain:
\begin{equation}
    \lim_{s\to 0} D(s) = \infty
\end{equation}
which ensures zero steady-state error for step velocity references and offers straightforward tuning in the frequency domain, 
making it highly suitable for motor velocity control.




\subsection{Design process and design parameters choices}

The controller parameters were obtained using the frequency-response method, 
based on the identified motor model $G_s(s)$ from Assignment 1, both for motor A and motor B.
The simplified continuous-time models after filtering are:
\begin{equation}
    G_{s,A}(s) = \frac{0.3677s^2 - 147.1s + 1.471e04}{s^2 + 237.1s + 7413}
\end{equation}
\begin{equation}
    G_{s,B}(s) = \frac{0.3763s^2 - 150.5s + 1.505e04}{s^2 + 237s + 7408}
\end{equation}


\paragraph{Choice of cross-over frequency $\omega_c$:} 
The cross-over frequency determines the closed-loop bandwidth. A value must balance:
\begin{itemize}
    \item Faster response vs. actuator limitations
    \item Tracking performance vs. noise sensitivity
    \item Stability margin vs. speed
\end{itemize}
A phase margin (PM) target of 55° was imposed to ensure robust behavior.
The uncompensated phase of the motor model was evaluated to find the frequency satisfying:
\begin{equation}
    \phi_G(\omega_c) = -180° + PM + \phi_{PI}(\omega_c)
\end{equation}
where:
\begin{itemize}
    \item $\phi_G(\omega_c)$ is the phase of the uncompensated open-loop system at $\omega_c$
    \item $PM$ is the desired phase margin, which is 55°
    \item $\phi_{PI}(\omega_c)$ is the anticipated phase lag of the PI compensator at $\omega_c$, which is selected to be 15°
\end{itemize}
This yields:
\begin{equation}
	\phi_G(\omega_c) = -180° + 55° + 15° = -110°
\end{equation}

The cross-over frequency $\omega_c$ where the phase of the uncompensated open-loop system equals -110° is fixed directly by the specification:
\begin{equation}
    \omega_{c,A} = XXX \texttt{rad/s}, \quad \omega_{c,B} = YYY \texttt{rad/s} 
    %\omega_{c,\text{low}} = 2\pi \cdot 0.5~\text{rad/s}~(0.50~\text{Hz})
\end{equation}


\paragraph{Integration time T$_i$:}
The integrator pole must lie well below the cross-over frequency to avoid excessive phase lag.
The integration time is selected such that the contribution of the PI compensator to the phase at $\omega_c$ equals the anticipated one ($\phi_{PI}(\omega_c)$ = 15°).
This design rule is mathematically expressed as:
\begin{equation}
    T_i \omega_c = \tan(90°-\phi_{PI}(\omega_c))
\end{equation}
using a margin of $\phi_{PI}(\omega_c)$ = 15°, leading to:
\begin{equation}
    T_{i,A} = XXX \texttt{s}, \quad T_{i,B} = YYY \texttt{s} 
	%T_{i,\text{low}} = 1.19~\text{s}
\end{equation}

Interpretation of trade-offs:
\begin{itemize}
    \item A smaller $T_i$ gives faster removal of steady-state error but adds more phase lag.
	\item A larger $T_i$ reduces phase lag but slows integral action.
\end{itemize}


\paragraph{Proportional gain $K$:} The gain is calculated such that the gain of the compensated system at $\omega_c$ equals 1:
\begin{equation}
    |D(j\omega_c)G(j\omega_c)| = 1
\end{equation}
The continuous controller is discretized using the Tustin method (with $c2d$ MATLAB command), 
and the gain is adjusted iteratively until MATLAB's Bode diagram confirms the desired cross-over frequency.
The identified wheel models yield the following proportional gains:
\begin{equation}
    K_A^{\text{nom}} = XXX, \quad K_B = YYY \quad
    %K_A^{\text{low}} = 0.489, \quad K_B^{\text{low}} = 0.477
\end{equation}

\paragraph{Design trade-offs:}
\begin{itemize}
	\item Increasing $PM$ provides robustness and damping but slows the response.
    \item Increasing $K$ increases bandwith but reduces phase margin ($PM$), raising the risk of oscillatory behavior and actuator saturation.
	\item Increasing $T_i$ reduces phase lag but slows convergence to steady-state (decreased integral speed).
	\item Increasing $\omega_c$ reduces delay sensitivity but amplifies encoder noise.
\end{itemize}

\paragraph{Verification requirements:}
The open-loop Bode plot $L(j\omega) = D(j\omega)G(j\omega)$ must show:
\begin{itemize}
	\item Unity gain ($|L(j\omega)| = 0 dB$) at $\omega_c$
	\item $PM$ ≈ 55°
	\item PI zero (at $\omega_z = \frac{1}{T_i}$) far below $\omega_c$
\end{itemize}
The time-domain verification must confirm:
\begin{itemize}
	\item Zero steady-state error on constant velocity reference
	\item Acceptable rise time ($t_r$ < 0.5s) and overshoot ($M_p$ < 20\%)
\end{itemize}



\paragraph{Bode Diagram Verification:}


Figure XXX shows the compensated open-loop system $L(j\omega) = D(j\omega)G(j\omega)$. 
Cross-over at $\omega_c$ with phase $\approx -125^\circ$ confirms PM $= 55^\circ$. Gain margin also indicated.
%
%\begin{figure}[h!]
%    \centering
%    \subfloat[Motor A]{\includegraphics[width=0.48\textwidth]{open_loop_bode_motorA}}\hfill
%    \subfloat[Motor B]{\includegraphics[width=0.48\textwidth]{open_loop_bode_motorB}}
%    \caption{Open-loop Bode diagram showing cross-over frequency $\omega_c$ and phase margin.}
%    \label{fig:openloop_bode}
%\end{figure}

Figure YYY shows the closed-loop system $H(j\omega) = \frac{L(j\omega)}{1+L(j\omega)}$. 
Bandwidth $\omega_{BW} \approx \omega_c$. Resonant peak due to finite PM. Gain rolls off beyond control bandwidth.

%\begin{figure}[h!]
%    \centering
%    \subfloat[Motor A]{\includegraphics[width=0.48\textwidth]{closed_loop_bode_motorA}}\hfill
%    \subfloat[Motor B]{\includegraphics[width=0.48\textwidth]{closed_loop_bode_motorB}}
%    \caption{Closed-loop Bode magnitude showing bandwidth $\omega_{BW}$.}
%    \label{fig:closedloop_bode}
%\end{figure}

\subsection{Theoretical and practical closed-loop bandwidth limitations}

\paragrpah{Theoretical limitation:} 
The bandwidth cannot exceed the frequency where the total open-loop phase $\phi_L$ approaches -180°. 
Beyond this point, increasing the gain drives the closed-loop system toward instability. 
For a first-order motor model, this typically limits cross-over frequency to approximately:
\begin{equation}
	\omega_c \approx \frac{1}{\tau}
\end{equation}
where $\tau$ is the motor time constant.

\paragraph{Practical limitations:} 

\begin{itemize}
    \item Sampling frequency (100 $Hz$): Nyquist limit of 50 $Hz$ (= $f_s/2$) cuts off the maximum achievable bandwidth. 
	\item Microcontroller delay introduces extra phase lag, reducing stability margins.
	\item Voltage saturation (≈ 12 $V$) limits achievable proportional gain.
	\item Encoder quantization noise leads to large high-frequency disturbances when bandwidth is too high.
	\item Motor friction and unmodeled nonlinearities reduce agreement with the designed frequency response.
\end{itemize}
These constraints limit the achievable closed-loop bandwidth even if theoretically higher values are possible.




\section{Experimental validation of the designed controller}


\subsection{Step response validation} % Section 2(a)



\paragraph{Experiment:} 
A constant velocity reference of 10 $rad/s$ is applied to both motors. The measured velocity is compared to the simulated velocity (using the compensated model).

\paragraph{Step response:}
The measured and simulated closed-loop responses exhibit similar transient patterns. Initial delays and rise times closely match the model predictions, confirming the validity of the identified motor dynamics. 
The steady-state velocity converges to the reference value of 10 $rad/s$, demonstrating correct integral action.
Oscillations in the measured signal may be slightly larger than in the simulation due to encoder quantization, unmodeled friction, and voltage saturation clipping.
Despite these differences, the measured response remains consistent with expected closed-loop behavior.

Figure XXX shows the step response comparison for both motors.
The step reference (dotted line), measured closed-loop response (solid line) and simulated closed-loop response (dashed line) are shown in one plot.

%\begin{figure}[h!]
%    \centering
%    \subfloat[Motor A]{\includegraphics[width=0.48\textwidth]{step_response_no_disturbance_motorA}}\hfill
%    \subfloat[Motor B]{\includegraphics[width=0.48\textwidth]{step_response_no_disturbance_motorB}}
%    \caption{Step response: reference (dotted) vs. measured (solid) vs. simulated (dashed).}
%    \label{fig:step_nodist}
%\end{figure}

\paragraph{Tracking error:} 
The initial tracking error matches the simulated transient and decays to approximately zero, 
apart from small steady-state fluctuations caused by sensor noise.

Figure YYY shows the tracking error comparison for both motors.
The measured (solid line) and simulated (dashed line) tracking error of the step reference are shown in one plot.

%\begin{figure}[h!]
%    \centering
%    \subfloat[Motor A]{\includegraphics[width=0.48\textwidth]{tracking_error_no_disturbance_motorA}}\hfill
%    \subfloat[Motor B]{\includegraphics[width=0.48\textwidth]{tracking_error_no_disturbance_motorB}}
%    \caption{Tracking error: measured (solid) vs. simulated (dashed).}
%    \label{fig:error_nodist}
%\end{figure}

\paragraph{Control signal (voltage):}
The control voltage initially spikes due to a large tracking error. The measured signal is clipped at roughly 12 $V$ due to hardware limits.
This clipping creates a slightly lower first overshoot compared to the simulation.
The control signal tracks the evolution of the velocity error as expected.

Figure ZZZ shows the control signal comparison for both motors.
The measured (solid line) and simulated (dashed line) control signal of the step reference are shown in one plot.

%\begin{figure}[h!]
%    \centering
%    \subfloat[Motor A]{\includegraphics[width=0.48\textwidth]{control_signal_no_disturbance_motorA}}\hfill
%    \subfloat[Motor B]{\includegraphics[width=0.48\textwidth]{control_signal_no_disturbance_motorB}}
%    \caption{Control voltage: measured (solid) vs. simulated (dashed). Saturation at ≈ 12 V is visible.}
%    \label{fig:control_nodist}
%\end{figure}

\paragraph{Performance characteristics:} 
Rise time ($t_r$), overshoot ($M_p$) and settling time ($t_s$) align with the design specifications.
The measured and simulated responses show closely matching dynamic properties.
The rise time $t_r$ of the measured system is slightly slower than in simulation due to unmodeled friction and voltage saturation, whereas the steady-state value is identical, confirming correct integral action.
The overshoot $M_p$ matches well in shape but is marginally larger in the experiment because encoder quantization and actuator limits introduce additional damping and asymmetry.
The settling time $t_s$ remains consistent between measurement and simulation, indicating that the designed phase margin ($PM$ = 55°) and bandwidth translate reliably to the real hardware.
Overall, the measured performance agrees with the designed characteristics: fast but well-damped convergence, full elimination of steady-state error, and control behavior that remains within actuator limits.


\subsection{Steady-state performance under constant force disturbance} % Section 2(b)


\paragraph{Experiment:} 
The cart is placed on an incline so that gravity (constant acceleration $g$ = 9.81 $m/s^2$) produces a constant negative torque, opposing the motion, on both wheels. 
The same velocity setpoint as in Section 2.1 is applied, namely a constant velocity reference of 10 $rad/s$.

\paragraph{Block diagram:} 
The constant force disturbance signal enters the loop as an external load torque acting on the motor dynamics. 
The disturbance affects the transient velocity, but the PI controller gradually compensates through its integral action.


Figure AAA shows the block diagram of the closed-loop system with the constant force disturbance.

%\begin{figure}[h!]
%    \centering
%    \includegraphics[width=0.48\textwidth]{block_diagram_disturbance}}
%    \caption{Block diagram of the closed-loop system with the constant force disturbance.}
%    \label{fig:block_diagram_disturbance}
%\end{figure}

\paragraph{Step response with disturbance:} 
The measured response shows lower peaks and a slower rise than the nominal case, 
while the simulated response remains unchanged because it does not include the disturbance.
Despite this mismatch in the transient, both measured and simulated curves converge to the same steady-state value due to the integrator, which compensates the constant load.

Figure XXX shows the step response comparison with disturbance for both motors.
The step reference (dotted line), measured closed-loop response (solid line) and simulated closed-loop response (dashed line) are shown in one plot.

%\begin{figure}[h!]
%    \centering
%    \subfloat[Motor A]{\includegraphics[width=0.48\textwidth]{step_response_with_disturbance_motorA}}\hfill
%    \subfloat[Motor B]{\includegraphics[width=0.48\textwidth]{step_response_with_disturbance_motorB}}
%    \caption{Step response with disturbance: reference (dotted) vs. measured (solid) vs. simulated (dashed).}
%    \label{fig:step_with_dist}
%\end{figure}

\paragraph{Tracking error with disturbance:} 
The measured tracking error is larger at the start and decays more smoothly, whereas the simulated error follows the original undisturbed transient.
Both measured and simulated tracking error eventually settle close to zero, but only the measured curve reflects the effect of the disturbance.

Figure YYY shows the tracking error comparison with disturbance for both motors.
The measured (solid line) and simulated (dashed line) tracking error of the step reference are shown in one plot.

%\begin{figure}[h!]
%    \centering
%    \subfloat[Motor A]{\includegraphics[width=0.48\textwidth]{tracking_error_with_disturbance_motorA}}\hfill
%    \subfloat[Motor B]{\includegraphics[width=0.48\textwidth]{tracking_error_with_disturbance_motorB}}
%    \caption{Tracking error with disturbance: measured (solid) vs. simulated (dashed).}
%    \label{fig:error_with_dist}
%\end{figure}

\paragraph{Control signal with disturbance:} 
The measured control signal settles at a visibly higher steady-state value than the simulated one, which remains unchanged without the disturbance term.
The transient shapes remain similar, but only the measured control effort compensates for the additional load.

Figure ZZZ shows the control signal comparison with disturbance for both motors.
The measured (solid line) and simulated (dashed line) control signal of the step reference are shown in one plot.

%\begin{figure}[h!]
%    \centering
%    \subfloat[Motor A]{\includegraphics[width=0.48\textwidth]{control_signal_with_disturbance_motorA}}\hfill
%    \subfloat[Motor B]{\includegraphics[width=0.48\textwidth]{control_signal_with_disturbance_motorB}}
%    \caption{Control voltage with disturbance: measured (solid) vs. simulated (dashed).}
%    \label{fig:control_with_dist}
%\end{figure}

\paragraph{Reference tracking despite disturbance:}
The controller continues to track the reference in steady state because the integral action of the PI controller compensates any constant disturbance by increasing the steady-state control effort until the velocity error becomes zero again.
Although the transient response is slowed down by the opposing force/torque, the integrator removes the steady-state error for constant disturbances, so the reference is still tracked.



\section{Controller validation with a cross-over frequency $\omega_c$ of 0.5 $Hz$ (including constant disturbance)} % Section 2(c)

\paragraph{Experiment:} 
With the redesigned PI controller, the cross-over frequency is reduced to about 0.5 $Hz$, 
which lowers the closed-loop bandwidth and yields a more conservative controller. 
The same constant disturbance (cart on a ramp) is applied and the plots of step response, tracking error and control signal now contain both controllers (high-bandwidth from Section 2.2 and low-bandwidth from Section 2.3).



\paragraph{Step response with low-bandwidth (and disturbance):}
The measured response with the low-bandwidth controller rises visibly slower and shows almost no overshoot, 
whereas the high-bandwidth controller from Section 2.2 reacts faster and exhibits more pronounced transient peaks. 
The simulated curves follow the same trend for each controller, but differences between measured and simulated responses remain slightly larger for the high-bandwidth case because it is more sensitive to model errors and disturbance effects.

Figure XXX shows the step response comparison for both motors. 
The step reference (dotted line), measured closed-loop response (solid line) and simulated closed-loop response (dashed line) are shown in one plot, both for the high-bandwidth and low-bandwidth case.

%\begin{figure}[h!]
%    \centering
%    \subfloat[Motor A]{\includegraphics[width=0.48\textwidth]{step_response_high_vs_low_BW_motorA}}\hfill
%    \subfloat[Motor B]{\includegraphics[width=0.48\textwidth]{step_response_high_vs_low_BW_motorB}}
%    \caption{Step response with low-bandwidth: reference (dotted) vs. measured (solid) vs. simulated (dashed).}
%    \label{fig:step_low_BW}
%\end{figure}


\paragraph{Tracking error with low-bandwidth (and disturbance):}
For the low-bandwidth controller the tracking error decreases more gradually with a smooth, almost aperiodic decay and small oscillations, while the high-bandwidth controller produces a quicker error reduction but with larger overshoot and ringing. 
In both cases the steady-state tracking error remains close to zero, confirming that the integral action compensates the constant disturbance.
The low-bandwidth controller, however, needs more time to eliminate the error.

Figure YYY shows the tracking error comparison for both motors.
The measured (solid line) and simulated (dashed line) tracking error of the step reference are shown in one plot, both for the high-bandwidth and low-bandwidth case.

%\begin{figure}[h!]
%    \centering
%    \subfloat[Motor A]{\includegraphics[width=0.48\textwidth]{tracking_error_high_vs_low_BW_motorA}}\hfill
%    \subfloat[Motor B]{\includegraphics[width=0.48\textwidth]{tracking_error_high_vs_low_BW_motorB}}
%    \caption{Tracking error with low-bandwidth: measured (solid) vs. simulated (dashed).}
%    \label{fig:error_low_BW}
%\end{figure}

\paragraph{Control signal with low-bandwidth (and disturbance):}
The measured control signal of the low-bandwidth controller has clearly smaller peaks and a smoother shape, 
but its steady-state voltage is comparable to or slightly higher than in Section 2.2, since the same disturbance torque must be balanced. 
The high-bandwidth controller generates sharper transients and larger initial control peaks, which push the actuator closer to saturation and amplify noise.
The simulated control signals show the same qualitative difference between both designs.

Figure ZZZ shows the control signal comparison for both motors.
The measured (solid line) and simulated (dashed line) control signal of the step reference are shown in one plot, both for the high-bandwidth and low-bandwidth case.

%\begin{figure}[h!]
%    \centering
%    \subfloat[Motor A]{\includegraphics[width=0.48\textwidth]{control_signal_high_vs_low_BW_motorA}}\hfill
%    \subfloat[Motor B]{\includegraphics[width=0.48\textwidth]{control_signal_high_vs_low_BW_motorB}}
%    \caption{Control voltage with low-bandwidth: measured (solid) vs. simulated (dashed).}
%    \label{fig:control_low_BW}
%\end{figure}

\paragraph{Contribution of the different (PI) controller parts:}
For the high-bandwidth controller, the proportional term (P) dominates the fast reaction to changes, 
resulting in rapid initial acceleration and stronger overshoot, while the integrator (I) mainly removes the residual steady-state error under disturbance. 
In the low-bandwidth design, the proportional gain $K$ is reduced and the cross-over frequency $\omega_c$ is lower, so the proportional action (P) reacts more slowly and the integrator (I) builds up more gradually.
This combination yields smoother, more damped responses with reduced peak control effort, at the cost of a slower convergence to the reference.


\bibliographystyle{plainnat}
\bibliography{references}

\end{document}
